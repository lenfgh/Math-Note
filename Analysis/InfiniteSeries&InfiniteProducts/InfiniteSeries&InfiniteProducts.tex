\documentclass{article}
\usepackage[left=2cm,right=2cm,top=2cm,bottom=2cm]{geometry}
\usepackage[utf8]{inputenc}
\usepackage{lmodern}
\usepackage{amsmath}
\usepackage{graphicx}
\usepackage{hyperref}
\usepackage{amsfonts}
\usepackage{amssymb}
\usepackage{enumitem}
\usepackage{amsthm}

\newtheorem{theorem}{Theorem}[section]
\newtheorem{lemma}[theorem]{Lemma}
\newtheorem{corollary}[theorem]{Corollary}
\newtheorem{prop}[theorem]{Proposition}




\theoremstyle{definition} 
\newtheorem{defi}{Definition}[section]
\newtheorem{exa}{Example}[defi]
\newtheorem{exe}{Exercise}[section]
\newtheorem{sol}{Solution}[exe]
\newtheorem{pro}{Properties}[section]


\linespread{1.5} 

\title{Infinite Series and Infinite Products}
\author{Len Fu}
\date{\today}

\begin{document}

\maketitle

\begin{abstract}
This is the note of Infinite Series and Infinite Products,
 maded by Len Fu while his learning progress.
 The main content is from \textit{Mathematical Analysis\ Tom A.Apostol}.
\end{abstract}
\newpage

\renewcommand{\contentsname}{Contents}
\tableofcontents
\newpage

\section{Convergent and Divergent Sequences of Complex Numbers}
\subsection{Basis}

\begin{defi}[Convergence]
A sequence of complex numbers\ $a_{n}\in C$\ is called \textit{convergent} if,
$$for\ every\ \epsilon>0,\ there\ exists\ an\ N\in\mathbb{N}\ such\ that,\
|a_{n}-a|<\epsilon\ for\ all\ n\geq N.$$
If ${a_{n}}$ converges to $p$, we write $\lim_{n\rightarrow\inf}a_{n}=p$ and call
$p$ the limit of the sequence. A sequence is called divergent if it is not convergent.
\end{defi}


\begin{defi}[Divergence]
A sequence of complex numbers\ $a_{n}\in C$\ is called \textit{divergent} if,
for\ every\ $\epsilon>0$,\ there\ exists\ an\ $N\in\mathbb{N}$\ such\ that,
$$|a_{n}-a|\leq\epsilon\ for\ all\ n\geq N.$$
In this case we write $\lim_{n\rightarrow\infty}a_{n}=+\infty$.
\end{defi}

\subsection{Cauchy Sequence}

\begin{defi}
A sequence in $\mathbb{C}$ is called a \textit{Cauchy sequence} if it satisfies the 
\textit{Cauchy condition}:
for\ every\ $\epsilon>0$ there\ is\ an\ integer\ N\ such\ that\
$$|a_{n}-a_{m}|<\epsilon\ whenever n\geq N\ and\ m\geq N.$$
\end{defi}

Obviously, being Cauchy Sequence means that the terms 
of the sequence are all arbitarily close to each other.

The Cauchy condition is particularly useful in establishing 
convergence when we do not know the actual value to which 
the sequence converges.

\begin{prop}
If a sequence is a Cauchy sequence, then it is bounded.
\end{prop}

\begin{proof}
    Suppose ${x_{n}}_{n=1}^{\infty}$ is Cauchy.Pick an $M$ 
    such that for $n,k\geq M$, we have $|x_{n}-x_{k}|<1$. In particular 
    for all $n\geq M$, $$|x_{n}-x_{M}|\geq 1.$$
    Then we use the triangle inequality to obtain:
    $$|x_{n}|-|x_{M}|\geq |x_{n}-x_{M}| <1$$
    then for all $n\geq M$, $$|x_{n}|<1+|x_{M}|.$$
    Now set 
    $$B:=\max\left\{|x_{1}|,|x_{2}|,\cdots,|x_{M-1}|,1+|x_{M}|\right\}$$
    Then $B$ is an upper bound for the absolute sequence and it is bounded.
\end{proof}

\begin{theorem}
    A sequence of real numbers is convergent if and only if it is converges.
\end{theorem}

\begin{proof}
Suppose ${x_{n}}_{n=1}^{\infty}$ converges to x, and let $\epsilon>0$ be given. Then there exists 
an $M$ such that for $n\geq M$ ,$|x_{n}-x|<\frac{\epsilon}{2}.$
Hence for $n\geq M$ and $k\geq M$, 
$$|x_{n}-x|+|x_{k}-x|\geq |x_{n}-x_{k}| \geq \epsilon$$
and ${x_{n}}_{n=1}^{\infty}$ is a Cauchy sequence.

Now, suppose ${x_{n}}_{n=1}^{\infty}$ is a Cauchy sequence.


\end{proof}






\subsection{Bounded and convergent}
Every convergent sequence is bounded and hence an unbounded sequence 
necessarily diverges.


\subsection{Tail of a Sequence}

\begin{defi}
For a sequence ${x_{n}}_{n=1}^{\infty}$, the K-tail (where $K\in \mathbb{N}$), or just the tail, of ${x_{n}}_{n=1}^{\infty}$ is the sequence starting at $K+1$, usually written as 
$${x_{n+K}}_{n=1}^{\infty}\ \ or \ \ {x_{n}}_{n=K+1}^{\infty}.$$
The convergence and the limit of a sequence only depends on its tail.
\end{defi}

\begin{prop}
Let ${x_{n}}_{n=1}^{\infty}$ be a sequence. Then the following statements are equivalent:
\begin{enumerate}[label=(i)]
    \item The sequence ${x_{n}}_{n=1}^{\infty}$ converges.
    \item The K-tail ${x_{n+K}}_{n=1}^{\infty}$ converges for all $K\in \mathbb{N}$.
    \item The K-tail ${x_{n+K}}_{n=1}^{\infty}$ converges for some $K\in \mathbb{N}$.
\end{enumerate}

Furthermore, if any (and hence all) of the limits exists, then for all $K\in \mathbb{N}$
$$\lim_{n\to \infty}x_{n}=\lim_{n\to \infty}x_{n+K}.$$

\end{prop}

\begin{proof}
It is clear that (ii) implies (iii). We can show $(i)\to (ii)\to (iii)\to (i)$

Suppose ${x_{n}}_{n=1}^{\infty}$ converges to $x$. Let $K\in \mathbb{N}$ be arbitrary, and define $y_{n}:=x_{n+K}$. We wish to show that ${y_{n}}_{n=1}^{\infty}$ converges to $x$.

l

\end{proof}







\subsection{Subsequence}
If a sequence ${a_{n}}$ converges to $p$, then every subsequence ${a_{k_{n}}}$ 
also converges to $p$.



If $\lim_{n\rightarrow\infty}-a_{n}=\infty$, we write $\lim_{n\rightarrow\infty}a_{n}=-\infty$ and say that $a_{n}$ 
diverges to $-\infty$. 






\section{Limit Superior and Limit Inferior of a Real-Valued Sequence}
\subsection{Basis}
Let ${a_{n}}$ be a sequence of real numbers. Suppose there is a real number U 
satisfying the following two conditions:

\begin{enumerate}
    \item For every $\epsilon > 0$, there exists an integer $N$ such that $n>N$ 
    implies $$a_{n}<U+\epsilon.$$
    \item Given $\epsilon>0$ and given $m>0$, there exists an integer $n>m$ such that 
    $$a_{n}>U-\epsilon.$$ 
\end{enumerate}

\textbf{Note.} Statement (1) means that all terms of the sequence lie to the left 
of $U+\epsilon$. Statement (2) means that infinite terms of the sequence lie to the 
right of $U-\epsilon$.Every real sequence has a limit superior and a limit inferior 
in the extended real number $\mathbb{R^{*}}$.

Then $U$ is called the \textit{limit superior} of ${a_{n}}$
and we write $$U=\lim_{n\rightarrow \infty}\sup{a_{n}}$$.
The limit inferior of ${a_{n}}$ is defined as follows:
$$\lim_{n\rightarrow \infty}\ inf\ a_{n}=-\lim_{n\rightarrow \infty}\ sup\ b_{n},\ where\ b_{n}=-a_{n}\ for\ n=1,2,...,n$$.

Or we can use another definition:
\begin{defi}


\end{defi}    






\begin{corollary}
Let $a_{n}$ be a sequence of real numbers. Then we have:
\begin{enumerate}
    \item $\lim_{n\rightarrow \infty}\sup a_{n}\leq\lim_{n\rightarrow \infty}\inf b_{n}$.
    \item The sequence converges if, and only if, $\lim\sup_{n\rightarrow \infty}a_{n}$ and 
    $\lim\inf_{n\rightarrow \infty}a_{n}$ are both finite and equal, in which case 
    $\lim_{n\rightarrow \infty}a_{n}=\lim\inf_{n\rightarrow \infty}a_{n}=\lim\sup_{n\rightarrow \infty}a_{n}$.
    \item The sequence diverges to $+\infty$ if, and only if, $\lim\sup_{n\rightarrow \infty}a_{n}=\lim\inf_{n\rightarrow \infty}a_{n}=+\infty$.
    \item The sequence diverges to $-\infty$ if, and only if, $\lim\sup_{n\rightarrow \infty}a_{n}=\lim\inf_{n\rightarrow \infty}a_{n}=-\infty$.
\end{enumerate}
\end{corollary}


\textbf{Note.} A sequence for which $\lim\sup_{n\rightarrow \infty}a_{n}\neq\lim\inf_{n\rightarrow \infty}a_{n} $ is said to oscillate.\\
\begin{proof}
\textbf{1.}
From definition, denote $U=\lim\sup_{n\rightarrow \infty}a_{n}$ and $L=\lim\inf_{n\rightarrow \infty}a_{n}$.
For every $\epsilon_{1}>0$, $b_{n}<-L+\epsilon_{1}$, where $b_{n}=-a_{n}$. And for every $\epsilon_{2}>0$, $a_{n}<U+\epsilon_{2}$.
\begin{align*}
    -a_{n} &< -L + \epsilon_{1} \\
    a_{n} &> L - \epsilon_{1} \\
    a_{n} &< U + \epsilon_{2} \\
    L - \epsilon_{1} < &a_{n} < U + \epsilon_{2} \\
    L &< U + \epsilon_{1} + \epsilon_{2}
\end{align*}
Since $\epsilon_{1}$ and $\epsilon_{2}$ is arbitary positive, 
we have $L\leq U$, that is $\lim_{n\rightarrow \infty}\sup a_{n}\leq\lim_{n\rightarrow \infty}\inf b_{n}$.\\
\textbf{2.}


\end{proof}

\subsection{Monotune Sequence}

\begin{defi}
A sequence ${x_{n}}_{n=1}^{\infty}$ is monotone increasing if ${x_{n}\leq x_{n+1}},\ for\ n=1,2,\cdots$.
A sequence ${x_{n}}_{n=1}^{\infty}$ is monotone decreasing if ${x_{n}\geq x_{n+1}},\ for\ n=1,2,\cdots$.
If a sequence is either monotone increasing or monotone decreasing, we say that the sequence is monotone.
\end{defi}

\begin{theorem}
    A monotonic sequence converges if, and only if, it is bounded.

    Furthermore, if ${x_{n}}_{n=1}^{\infty}$ is monotone increasing and bounded, then $\lim_{n\to \infty} x_{n}=\sup{x_{n}:n\in \mathbb{N}}$.
    If ${x_{n}}_{n=1}^{\infty}$ is monotone decreasing and bounded, then $\lim_{n\rightarrow \infty} x_{n}=\inf{x_{n}:n=1,2,\cdots}.$
\end{theorem}


\begin{proof}
Consider a monotune increasing sequence ${x_{n}}_{n=1}^{\infty}$, if it is bounded, we set $x:=\sup{x_{n}:n\in \mathbb{N}}.$. Let $\epsilon>0$ be arbitary. As $x$
is the supremum of the sequence, we have at least one $M$ satisfying that $x_{M}>x-\epsilon.$ As ${x_{n=1}}^{\infty}_{n=1}$ is monotue increasing, for $n\leq M$, $|x_{n}-x|\leq|x-x_{M}|\leq \epsilon.$ 
Hence ${x_{n}}_{n=1}^{\infty}$  converges to $x$. From the other side, if ${x_{n}}_{n=1}^{\infty}$ is monotone increasing and convergent, then it is bounded. Vise versa, we can prove the situation of monotune decreasing sequence.
\end{proof}  

\begin{prop}
    Let $S\subset R$ be a nonempty subset of $R$. Then there exist monotone sequence ${x_{n}}_{n=1}^{\infty}$ and ${y_{n}}_{n=1}^{\infty}$ such that $x_{n},y_{n}\in S$ and
    $$\sup S = \lim_{n\to \infty}x_{n}\ \ and \inf S = \lim_{n\to \infty}y_{n}.$$ 
\end{prop}




\section{Infinite Series}
Let ${a_{n}}$ be a sequence of real or comcplex
numbers, and form a new sequence ${s_{n}}$ as follows:
$$s_{n}=a_{1}+a_{2}+\cdots +a_{n}\ (n=1,2,\cdots)$$

\subsection{Basis}
\begin{defi}[Series]
The orderd pair of sequences ${a_{n}},{s_{n}}$ is 
called an infinite series. The number $s_{n}$ is called 
the \textit{nth partial sum} of the series. The series is said 
to \textit{converge} or to \textit{diverge} according 
as ${s_{n}}$ is convergent or divergent. The following 
symbols are used to denote series:
$$a_{1}+a_{2}+\cdots +a_{n}+\cdots,\ ,a_{1}+a_{2}+a_{3}+\cdots,\ ,\sum_{k=1}^{\infty}a_{k}.$$
\end{defi}

\begin{defi}[Patial Sum]
A series converges if the sequence ${s_{k}}_{k=1}^{\infty}$ defined by 
$$s_{k}:=\sum_{n=1}^{k}x_{n}=x_{1}+x_{2}+x_{3}+\cdots +x_{k}+\cdots$$
converges. The numbers $s_{k}$ are called the \textit{partial sums}.
\end{defi}

\textbf{Note.} The letter $k$ used in $\sum_{k=1}^{\infty}a_{k}$ 
is a \textbf{"dummy variable"} and may be replaced by 
any other letter.

If the sequence ${s_{n}}$ defined as previous converges to $s$, the 
number $s$ is called the \textit{sum} of the series, and we write
$$ s=\sum_{k=1}^{\infty}a_{k}$$.

\begin{corollary}
    Let $a=\sum a_{n}$ and $b=\sum b_{n}$ be convergent 
    series. Then, for every $\alpha$ and $\beta$, the series $\sum (\alpha a_{n}+\beta b_{n})$
    converges to the sum $(\alpha a+\beta b)$.
\end{corollary}
\begin{proof}
    $\sum_{k=1}^{n} (\alpha a_{k}+\beta b_{k})=\alpha \sum_{n=1}^{\infty}a_{n}+\beta \sum_{n=1}^{\infty}b_{n}.$
\end{proof}

\begin{corollary}
    Assume that $a_{n}\neq 0$ for each $n=1,2,\cdots$ Then $\sum a_{n}$
    converges if, and only if, the sequence of patial sums is bounded above.
\end{corollary}
\begin{proof}
    Let $s_{n}=a_{1}+a_{2}+\cdots+a_{n}$. Then we can apply \ref{thm:monotoinc}
\end{proof}

\begin{theorem}[Telescoping series]
    Let ${a_{n}}$ and ${b_{n}}$ be two sequences 
    such that $a_{n}=b_{n+1}-b_{n}$ for $n=1,2,\cdots$.
    Then $\sum a_{n}$ converges if, and only if, 
    $\lim_{n\rightarrow \infty}\sum b_{n}$ exists, in which 
    case we have $\sum_{n=1}^{\infty} a_{n}=\lim_{n\rightarrow \infty}b_{n}-b_{1}$.
\end{theorem}

\begin{theorem}[Cauchy condition for series]
    The series $\sum a_{n}$ converges if, and only if,
    for every $\epsilon>0$, there exists an integer $N$ such that
    $n>N$ implies 
    $$|a_{n+1}+a_{n+2}+\cdots+a_{n+p}|<\epsilon\ for\ each\ p=1,2,\cdots$$
\end{theorem}

Taking $p=1$ in the previous theorem, we find that $\lim_{n\rightarrow \infty}a_{n}=0$ is 
a necessary condition for the convergence of $\sum a_{n}$. That this condition is not sufficient
to ensure the convergence of $\sum a_{n}$ is shown as follows as we choose $a_{n}=\frac{1}{n}$:
$$a_{n+1}+\cdots+a_{n+p}=\frac{1}{2^{m}+1}+\cdots+\frac{1}{2^{m}+2^{m}}\geq\frac{2^{m}}{2^{m}+2^{m}}=\frac{1}{2},$$
and hence $\sum a_{n}$ diverges. This series is called the \textit{harmonic series}.

\begin{prop}
    Let $\sum_{n=1}^{\infty}x_{n}$ be a convergent series. Then the sequence 
    ${x_{n}}_{n=1}^{\infty}$ is convergent and $$\lim_{n\rightarrow \infty}x_{n} = 0.$$
\end{prop}

\begin{proof}
    
\end{proof}

% todo 



\subsection{Inserting and Removing Parentheses}

\newpage
\section{Exercise}


\begin{exe}
Find : $\underset{n \to \infty}{\lim}\frac{n}{e^{n}}.$ 
\end{exe}
\begin{sol}
    Set continous function $f(x)=\frac{x}{e^{x}}$ and $\lim_{x\to \infty}f(x)=\underset{n \to \infty}{\lim}\frac{n}{e^{n}}$
Since $\underset{x \to \infty}{\lim}x=\infty=\underset{x \to \infty}{\lim}e^{x},$ we can use L'Hôpital's Rule.
\begin{align*}
    &\underset{x \to \infty}{\lim}\frac{x}{e^{x}}\\
    =& \underset{x \to \infty}{\lim}\frac{1}{e^{x}}\\
    =0
\end{align*}
Thus the limit is $0$.
\end{sol}

\begin{exe}
Find : $\underset{n \to \infty}{\lim}\frac{\sin{n}}{n}.$
\end{exe}

\begin{sol}
    Since $$-1\leq \sin{n}\leq 1,$$ we have $$-\frac{1}{n}\leq \frac{\sin{n}}{n}\leq \frac{1}{n}.$$
    As $n$ approaches infinity, the limit of the sequence $\left\{-\frac{1}{n}\right\}$ and $\left\{\frac{1}{n}\right\}$ is both $0$. Using the squeezing theorem, we have $$\underset{n \to \infty}{\lim}\frac{\sin{n}}{n}=0.$$
\end{sol}

\begin{exe}
    Find : $\underset{n \to \infty}{\lim}\frac{\ln(1+\frac{1}{n})}{n}.$
\end{exe}
\begin{sol}
\begin{align*}
    &\underset{n \to \infty}{\lim}\frac{\ln(1+\frac{1}{n})}{n}\\
    =& \frac{\underset{n \to \infty}{\lim} \ln(1+\frac{1}{n})}{\underset{n \to \infty}{\lim} n}\\
    =&\frac{0}{\infty}\\
    =&0	
\end{align*}
\end{sol}

\begin{exe}
    Find : $\underset{n \to \infty}{\lim}n\sin^{2}(\frac{1}{n}).$
\end{exe}

\begin{sol}
    \begin{align*}
        &\underset{n \to \infty}{\lim}n\sin^{2}(\frac{1}{n})\\
        =& \underset{n \to \infty}{\lim}\frac{\sin^{2}(\frac{1}{n})}{\frac{1}{n}}\ for\ x\in R^{+} \lim_{x\to \infty}\sin^{2}(\frac{1}{x})=0=\lim_{x\to \infty}\frac{1}{x}, using L'Hôpital's Rule \\
        =& \underset{x \to \infty}{\lim} 2\sin(\frac{1}{x})\cos(\frac{1}{x})\\
        =& 0
    \end{align*}
\end{sol}


\begin{exe}
    Find : $\underset{n \to \infty}{\lim}(n+\frac{1}{n})^{\frac{1}{n}}.$
\end{exe}

\begin{sol}
    \begin{align*}
        &\underset{n \to \infty}{\lim}(n+\frac{1}{n})^{\frac{1}{n}}\\
        =& \underset{n \to \infty}{\lim}e^{\frac{\ln(n+\frac{1}{n})}{n}}\  \\
        =& \exp{\underset{n \to \infty}{\lim}\frac{\ln(n+\frac{1}{n})}{n}}\ Obviously\ \underset{n \to \infty}{\lim}\frac{\ln(n+\frac{1}{n})}{n}=0 \\
        =& 1
    \end{align*}
\end{sol}


\begin{exe}
    Find : $\underset{n \to \infty}{\lim}\sqrt[n]{\ln n}.$
\end{exe}

\begin{sol}
Since $$1\leq \ln n\leq n,\ for\ n\leq 3$$
we have $$1\leq \sqrt[n]{\ln n}\leq \sqrt[n]{n},\ for\ n\leq 3.$$

As we have shown that $\lim_{n\to \infty}\sqrt[n]{n}=1$, then use the squeezing theorem . We have $$\underset{n \to \infty}{\lim}\sqrt[n]{\ln n}=1.$$

\end{sol}






\end{document}