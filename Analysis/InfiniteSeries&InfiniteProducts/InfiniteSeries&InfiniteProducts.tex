\documentclass{article}
\usepackage[left=2cm,right=2cm,top=2cm,bottom=2cm]{geometry}
\usepackage[utf8]{inputenc}
\usepackage{lmodern}
\usepackage{amsmath}
\usepackage{graphicx}
\usepackage{hyperref}
\usepackage{amsfonts}
\usepackage{amssymb}
\usepackage{enumitem}
\usepackage{amsthm}

\newtheorem{theorem}{Theorem}[section]
\newtheorem{lemma}[theorem]{Lemma}
\newtheorem{corollary}[theorem]{Corollary}
\newtheorem{prop}[theorem]{Proposition}




\theoremstyle{definition} 
\newtheorem{defi}{Definition}[section]
\newtheorem{exa}{Example}[defi]
\newtheorem{exe}{Exercise}[section]
\newtheorem{sol}{Solution}[exe]
\newtheorem{pro}{Properties}[section]


\linespread{1.5} 

\title{Infinite Series and Infinite Products}
\author{Len Fu}
\date{\today}

\begin{document}

\maketitle

\begin{abstract}
This is the note of Infinite Series and Infinite Products,
 maded by Len Fu while his learning progress.
 The main content is from \textit{Mathematical Analysis\ Tom A.Apostol} and the course of \textit{Mathematical Analysis\ Z.H.Zhao BIT 2024 Fall}.

In this chapter,  you need to 
\end{abstract}
\newpage

\renewcommand{\contentsname}{Contents}
\tableofcontents
\newpage



\section{Convergence and Divergence}

\textbf{Completeness Axiom of Real Numbers:} The real numbers satisfy the following completeness axioms:

\begin{enumerate}
    \item \textbf{Least Upper Bound Property:} Every non-empty subset of \(\mathbb{R}\) that is bounded above has a least upper bound (supremum) in \(\mathbb{R}\).
    \item \textbf{Greatest Lower Bound Property:} Every non-empty subset of \(\mathbb{R}\) that is bounded below has a greatest lower bound (infimum) in \(\mathbb{R}\).
    \item \textbf{Nested Interval Theorem:} Let \(I_n = [a_n, b_n]\) be a sequence of closed intervals such that \(I_1 \supseteq I_2 \supseteq I_3 \supseteq \ldots\) and \(\lim_{n \to \infty} (b_n - a_n) = 0\). Then, the intersection of all these intervals, denoted as \(\bigcap_{n=1}^{\infty} I_n\), consists of exactly one real number \(x\).
    \item \textbf{Bolzano-Weierstrass Theorem:} Every bounded sequence of real numbers has a convergent subsequence.
    \item \textbf{Cauchy Convergence Criterion:} A sequence of real numbers converges if and only if it is a Cauchy sequence.
\end{enumerate}


\subsection{Basis}

\begin{defi}[Convergence]
A sequence of complex numbers\ $a_{n}\in C$\ is called \textit{convergent} if,
$$for\ every\ \epsilon>0,\ there\ exists\ an\ N\in\mathbb{N}\ such\ that,\
|a_{n}-a|<\epsilon\ for\ all\ n\geq N.$$
If ${a_{n}}$ converges to $p$, we write $\lim_{n\rightarrow\inf}a_{n}=p$ and call
$p$ the limit of the sequence. A sequence is called divergent if it is not convergent.
\end{defi}


\begin{defi}[Divergence]
A sequence of complex numbers\ $a_{n}\in C$\ is called \textit{divergent} if,
for\ every\ $\epsilon>0$,\ there\ exists\ an\ $N\in\mathbb{N}$\ such\ that,
$$|a_{n}-a|\leq\epsilon\ for\ all\ n\geq N.$$
In this case we write $\lim_{n\rightarrow\infty}a_{n}=+\infty$.
\end{defi}

\subsection{Cauchy Sequence}

\begin{defi}
A sequence in $\mathbb{C}$ is called a \textit{Cauchy sequence} if it satisfies the 
\textit{Cauchy condition}:
for\ every\ $\epsilon>0$ there\ is\ an\ integer\ N\ such\ that\
$$|a_{n}-a_{m}|<\epsilon\ whenever n\geq N\ and\ m\geq N.$$
\end{defi}

Obviously, being Cauchy Sequence means that the terms 
of the sequence are all arbitarily close to each other.

The Cauchy condition is particularly useful in establishing 
convergence when we do not know the actual value to which 
the sequence converges.

\begin{prop}
If a sequence is a Cauchy sequence, then it is bounded.
\end{prop}

\begin{proof}
    Suppose $\left\{x_{n}\right\}_{n=1}^{\infty}$ is Cauchy.Pick an $M$ 
    such that for $n,k\geq M$, we have $|x_{n}-x_{k}|<1$. In particular 
    for all $n\geq M$, $$|x_{n}-x_{M}|\geq 1.$$
    Then we use the triangle inequality to obtain:
    $$|x_{n}|-|x_{M}|\geq |x_{n}-x_{M}| <1$$
    then for all $n\geq M$, $$|x_{n}|<1+|x_{M}|.$$
    Now set 
    $$B:=\max\left\{|x_{1}|,|x_{2}|,\cdots,|x_{M-1}|,1+|x_{M}|\right\}$$
    Then $B$ is an upper bound for the absolute sequence and it is bounded.
\end{proof}

\begin{theorem}
    A sequence of real numbers is convergent if and only if it is converges.
\end{theorem}

\begin{proof}
Suppose $\left\{x_{n}\right\}_{n=1}^{\infty}$ converges to x, and let $\epsilon>0$ be given. Then there exists 
an $M$ such that for $n\geq M$ ,$|x_{n}-x|<\frac{\epsilon}{2}.$
Hence for $n\geq M$ and $k\geq M$, 
$$|x_{n}-x|+|x_{k}-x|\geq |x_{n}-x_{k}| \geq \epsilon$$
and $\left\{x_{n}\right\}_{n=1}^{\infty}$ is a Cauchy sequence.

Now, suppose $\left\{x_{n}\right\}_{n=1}^{\infty}$ is a Cauchy sequence.
\end{proof}


Actually, every convergent sequence is bounded and hence an unbounded sequence necessarily diverges.


\subsection{Tail of a Sequence}

\begin{defi}
For a sequence $\left\{x_{n}\right\}_{n=1}^{\infty}$, the K-tail (where $K\in \mathbb{N}$), or just the tail, of $\left\{x_{n}\right\}_{n=1}^{\infty}$ is the sequence starting at $K+1$, usually written as 
$${x_{n+K}}_{n=1}^{\infty}\ \ or \ \ {x_{n}}_{n=K+1}^{\infty}.$$
The convergence and the limit of a sequence only depends on its tail.
\end{defi}

\begin{prop}
Let $\left\{x_{n}\right\}_{n=1}^{\infty}$ be a sequence. Then the following statements are equivalent:
\begin{enumerate}[label=(i)]
    \item The sequence $\left\{x_{n}\right\}_{n=1}^{\infty}$ converges.
    \item The K-tail ${x_{n+K}}_{n=1}^{\infty}$ converges for all $K\in \mathbb{N}$.
    \item The K-tail ${x_{n+K}}_{n=1}^{\infty}$ converges for some $K\in \mathbb{N}$.
\end{enumerate}

Furthermore, if any (and hence all) of the limits exists, then for all $K\in \mathbb{N}$
$$\lim_{n\to \infty}x_{n}=\lim_{n\to \infty}x_{n+K}.$$

\end{prop}

\begin{proof}
It is clear that (ii) implies (iii). We can show $(i)\to (ii)\to (iii)\to (i)$

Suppose $\left\{x_{n}\right\}_{n=1}^{\infty}$ converges to $x$. Let $K\in \mathbb{N}$ be arbitrary, and define $y_{n}:=x_{n+K}$. We wish to show that ${y_{n}}_{n=1}^{\infty}$ converges to $x$.
\end{proof}

\subsection{Recursively Defined Sequence}

One such class are recursively defined sequences are that the next term depends on the previous term and have a fixed formula.

To prove a recursively defined sequence is a 



\subsection{Subsequence}

\begin{defi}
    Let $\left\{x_{n}\right\}_{n=1}^{\infty}$ be a squence. Let ${n_{i}}_{i=1}^{\infty}$ be a strictly increasing sequence of integers.Then the sequence 
    $$\left\{x_{n_{i}}\right\}_{i=1}^{\infty}$$ is called a \textit{subsequence} of $\left\{x_{n}\right\}_{n=1}^{\infty}$.
\end{defi}


\begin{prop}
    If $\left\{x_{n}\right\}_{n=1}^{\infty}$ is a convergent sequence, then every subsequence ${x_{n_{i}}}_[i=1]^{\infty}$ is convergent, and 
    $$\lim_{n\to \infty}x_{n}=\lim_{i\to \infty}x_{n_{i}}.$$ 
    That is, if a sequence ${a_{n}}$ converges to $p$, then every subsequence ${a_{k_{n}}}$ also converges to $p$.

\end{prop}



If $\lim_{n\rightarrow\infty}-a_{n}=\infty$, we write $\lim_{n\rightarrow\infty}a_{n}=-\infty$ and say that $a_{n}$ 
diverges to $-\infty$. 


\subsection{About the Limit of a Sequence}

\begin{theorem}[Relationship between the Limit of a Sequence and the Limit of a Function]
    Let $\left\{a_{n}\right\}_{n=1}^{\infty}$ be a sequence, and $f$ is a function defined on $[m,+\infty)$ such that $f(n)=a_{n}\ for\ n\geq m.$
    \begin{enumerate}
        \item If $\lim_{x\to\infty}f(x)=L$, then $\left\{a_{n}\right\}_{n=1}^{\infty}$ is convergent and $\lim_{n\rightarrow\infty}a_{n}=L$.
        \item If $\lim_{x\to\infty}f(x)=\pm \infty$, then $\left\{a_{n}\right\}_{n=1}^{\infty}$ is divergent and $\lim_{n\rightarrow\infty}a_{n}=\pm \infty$.
    \end{enumerate}
\end{theorem}


\begin{lemma}[Squezze Theorem]

    Let $\left\{a_{n}\right\}_{n=1}^{\infty}$ and $\left\{b_{n}\right\}_{n=1}^{\infty}$ be sequences of real numbers, and $\left\{x_{n}\right\}_{n=1}^{\infty}$
     be a sequence satisfying 
     $$a_{n}<x_{n}<b_{n}\ for\ all\ n\in\mathbb{N}.$$

    Suppose that $\lim_{n\rightarrow\infty}a_{n}=\lim_{n\rightarrow\infty}b_{n}=$,
    then $\lim_{n\rightarrow\infty}a_{n}\lim_{n\rightarrow\infty}x_{n}=\lim_{n\rightarrow\infty}b_{n}$.
    
\end{lemma}

\begin{lemma}
    
    Let $\left\{a_{n}\right\}_{n=1}^{\infty}$ and $\left\{b_{n}\right\}_{n=1}^{\infty}$ be convergent sequences and 
    $$a_{n}\leq b_{n}\ for\ all\ n\in \mathbb{N}.$$
    Then $\lim_{n\rightarrow\infty}a_{n}\leq \lim_{n\rightarrow\infty}b_{n}$.
\end{lemma}

\begin{corollary}
    \begin{enumerate}
        \item If $\left\{x_{n}\right\}_{n=1}^{\infty}$ is a convergent sequence such that $x_{n}\leq 0$ for all $n\in \mathbb{N}$, then $\lim_{n\to\infty}x_{n}\geq0$.
        \item Let $a,b\in\mathbb{R}$ and let $\left\{x_{n}\right\}_{n=1}^{\infty}$ be a convergent sequence such that 
        $$a\leq x_{n}\leq b\ for\ all\ n\in \mathbb{N}.$$
        Then $a\leq \lim_{n\rightarrow\infty}x_{n}\leq b$.
    \end{enumerate}
\end{corollary}

\begin{prop}[Algebraic Operations]

    Let $\left\{x_{n}\right\}_{n=1}^{\infty}$ and $\left\{y_{n}\right\}_{n=1}^{\infty}$ be convergent sequences. Then the following statements are true:
\begin{enumerate}
    \item For $z_{n}:=x_{n}+y_{n}$, it converges and $$\lim_{n\rightarrow \infty}z_{n}=\lim_{n\rightarrow \infty}x_{n}+\lim_{n\rightarrow \infty}y_{n}=\lim_{n\rightarrow \infty}(x_{n}+y_{n}).$$
    \item For $z_{n}:=x_{n}-y_{n}$, it converges and $$\lim_{n\rightarrow \infty}z_{n}=\lim_{n\rightarrow \infty}x_{n}-\lim_{n\rightarrow \infty}y_{n}=\lim_{n\rightarrow \infty}(x_{n}-y_{n}).$$
    \item For $z_{n}:=x_{n}y_{n}$, it converges and $$\lim_{n\rightarrow \infty}z_{n}=\lim_{n\rightarrow \infty}x_{n}\lim_{n\rightarrow \infty}y_{n}=\lim_{n\rightarrow \infty}(x_{n}y_{n}).$$
    \item For $z_{n}:=\frac{x_{n}}{y_{n}}$, if $\lim_{n\to \infty} y_{n}\neq 0 and y_{n}\neq 0$,it converges and $$\lim_{n\rightarrow \infty}z_{n}=\lim_{n\rightarrow \infty}\frac{x_{n}}{y_{n}}=\lim_{n\rightarrow \infty}\left(\frac{x_{n}}{y_{n}}\right).$$
\end{enumerate}
\end{prop}


\subsection{Convergence Test}
\begin{lemma}[Ratio test for sequences]
    Let $\left\{a_{n}\right\}_{n=1}^{\infty}$ be a sequence of real numbers such that $a_{n}\neq 0$ for all $n\in \mathbb{N}$. If the limit 
    $$L=\lim_{n\to \infty}\frac{|x_{n+1}|}{|x_{n}|}\ exists.$$
    And 
    \begin{enumerate}
        \item If $L<1$, then $\left\{a_{n}\right\}_{n=1}^{\infty}$ converges to 0.
        \item If $L>1$, then $\left\{a_{n}\right\}_{n=1}^{\infty}$ diverges.
    \end{enumerate}
\end{lemma}

\begin{theorem}[Boundedness]
    \begin{enumerate}
        \item If $\left\{a_{n}\right\}_{n=1}^{\infty}$ converges, then it is bounded.
        \item If $\left\{a_{n}\right\}_{n=1}^{\infty}$ is unbounded, then it diverges.
    \end{enumerate}
\end{theorem}
\noindent\textbf{Note:} It does not imply that all bounded sequences converge. 

\subsection{Important Theorems}
\begin{theorem}[Bolzano-Weierstrass Theorem]
    Let $\left\{a_{n}\right\}_{n=1}^{\infty}$ be a convergent sequence of real numbers. Then there exists a convergent subsequence $\left\{x_{n_{i}}\right\}_{i=1}^{\infty}$.
\end{theorem}

\begin{theorem}[Nested Interval Theorem]
    Let \( I_n = [a_n, b_n] \) be a sequence of closed intervals such that:
\begin{enumerate}
    \item \( I_1 \supseteq I_2 \supseteq I_3 \supseteq \ldots \) (the intervals are nested),
    \item \( \lim_{n \to \infty} (b_n - a_n) = 0 \) (the length of the intervals converges to 0).
\end{enumerate}
Then, the intersection of all these intervals, denoted as \( \bigcap_{n=1}^{\infty} I_n \), consists of exactly one real number \( x \). Moreover, both sequences \( \{a_n\} \) and \( \{b_n\} \) converge to \( x \).
\end{theorem}



\section{Limit Superior and Limit Inferior of a Real-Valued Sequence}
\subsection{Basis}
Let ${a_{n}}$ be a sequence of real numbers. Suppose there is a real number U 
satisfying the following two conditions:

\begin{enumerate}
    \item For every $\epsilon > 0$, there exists an integer $N$ such that $n>N$ 
    implies $$a_{n}<U+\epsilon.$$
    \item Given $\epsilon>0$ and given $m>0$, there exists an integer $n>m$ such that 
    $$a_{n}>U-\epsilon.$$ 
\end{enumerate}

\textbf{Note.} Statement (1) means that all terms of the sequence lie to the left 
of $U+\epsilon$. Statement (2) means that infinite terms of the sequence lie to the 
right of $U-\epsilon$.Every real sequence has a limit superior and a limit inferior 
in the extended real number $\mathbb{R^{*}}$.

Then $U$ is called the \textit{limit superior} of ${a_{n}}$
and we write $$U=\lim_{n\rightarrow \infty}\sup{a_{n}}$$.
The limit inferior of ${a_{n}}$ is defined as follows:
$$\lim_{n\rightarrow \infty}\ inf\ a_{n}=-\lim_{n\rightarrow \infty}\ sup\ b_{n},\ where\ b_{n}=-a_{n}\ for\ n=1,2,...,n$$.

\begin{corollary}
    Let $a_{n}$ be a sequence of real numbers. Then we have:
    \begin{enumerate}
        \item $\lim_{n\rightarrow \infty}\sup a_{n}\leq\lim_{n\rightarrow \infty}\inf b_{n}$.
        \item The sequence converges if, and only if, $\lim\sup_{n\rightarrow \infty}a_{n}$ and 
        $\lim\inf_{n\rightarrow \infty}a_{n}$ are both finite and equal, in which case 
        $\lim_{n\rightarrow \infty}a_{n}=\lim\inf_{n\rightarrow \infty}a_{n}=\lim\sup_{n\rightarrow \infty}a_{n}$.
        \item The sequence diverges to $+\infty$ if, and only if, $\lim\sup_{n\rightarrow \infty}a_{n}=\lim\inf_{n\rightarrow \infty}a_{n}=+\infty$.
        \item The sequence diverges to $-\infty$ if, and only if, $\lim\sup_{n\rightarrow \infty}a_{n}=\lim\inf_{n\rightarrow \infty}a_{n}=-\infty$.
    \end{enumerate}
    \end{corollary}
    
    
    \textbf{Note.} A sequence for which $\lim\sup_{n\rightarrow \infty}a_{n}\neq\lim\inf_{n\rightarrow \infty}a_{n} $ is said to oscillate.\\
    \begin{proof}
    \textbf{1.}
    From definition, denote $U=\lim\sup_{n\rightarrow \infty}a_{n}$ and $L=\lim\inf_{n\rightarrow \infty}a_{n}$.
    For every $\epsilon_{1}>0$, $b_{n}<-L+\epsilon_{1}$, where $b_{n}=-a_{n}$. And for every $\epsilon_{2}>0$, $a_{n}<U+\epsilon_{2}$.
    \begin{align*}
        -a_{n} &< -L + \epsilon_{1} \\
        a_{n} &> L - \epsilon_{1} \\
        a_{n} &< U + \epsilon_{2} \\
        L - \epsilon_{1} < &a_{n} < U + \epsilon_{2} \\
        L &< U + \epsilon_{1} + \epsilon_{2}
    \end{align*}
    Since $\epsilon_{1}$ and $\epsilon_{2}$ is arbitary positive, 
    we have $L\leq U$, that is $\lim_{n\rightarrow \infty}\sup a_{n}\leq\lim_{n\rightarrow \infty}\inf b_{n}$.\\
    \textbf{2.}
    
    \end{proof}



We have another definition for \textbf{Bounded Sequence}:
\begin{defi}
Let $\left\{x_{n}\right\}_{n=1}^{\infty}$ be a bounded sequence. Define $\left\{a_{n}\right\}_{n=1}^{\infty}$ and $\left\{b_{n}\right\}_{n=1}^{\infty}$ by $a_{n}:=sup{x_{k}:k\geq n}$ and 
$b_{n}:=inf{x_{k}:k\geq n}$. Define, if the limits exist,
$$\lim_{n\rightarrow \infty}\sup x_{n} =\lim_{n\rightarrow \infty} a_{n}\ and\ \lim_{n\rightarrow \infty}\inf x_{n} =\lim_{n\rightarrow \infty} b_{n}$$
\end{defi}    

\begin{prop}
    \begin{enumerate}
        \item The sequence $\left\{a_{n}\right\}_{n=1}^{\infty}$ is bounded monotone decreasing and 
        $\left\{b_{n}\right\}_{n=1}^{\infty}$ is bounded monotone increasing.
        \item $\lim_{n\rightarrow \infty}\sup x_{n}=\inf{a_{n}:n\in\mathbb{N}}$ and $\lim_{n\to\infty}\inf x_{n}=\sup{b_{n}:n\in\mathbb{N}}$.
        \item $\lim_{n\to\infty}\inf x_{n}\leq \lim_{n\to\infty}\sup x_{n}$.
    \end{enumerate}
\end{prop}

\begin{theorem}
    If $\left\{x_{n}\right\}_{n=1}^{\infty}$ is a bounded sequence, then there exists a subsequence 
    $\left\{x_{n_{k}}\right\}_{k=1}^{\infty}$ such that 
    $$\lim_{k\to\infty}x_{n_{k}}=\lim_{n\to\infty}\sup x_{n}.$$
    Similarly, there exists a subsequence $\left\{x_{m_{k}}\right\}_{k=1}^{\infty}$ such that
    $$\lim_{k\to\infty}x_{m_{k}}=\lim_{n\to\infty}\inf x_{n}.$$
\end{theorem}

\begin{prop}
    For a bounded sequence $\left\{x_{n}\right\}_{n=1}^{\infty}$, if it is convergent, then 
    $$\lim_{n\to\infty} \inf x_{n}=\lim_{n\to\infty} \sup x_{n}.$$
    Furthermore, if $\left\{x_{n}\right\}_{n=1}^{\infty}$ converges, then 
    $$\lim_{n\to\infty} x_{n}=\lim_{n\to\infty} \inf x_{n} = \lim_{n\to\infty} \sup x_{n}.$$
\end{prop}

\begin{prop}
Suppose $\left\{x_{n}\right\}_{n=1}^{\infty}$ is a bounded sequence and $\left\{x_{k}\right\}_{k=1}^{\infty}$ is a subsequence.
Then 
$$\lim_{n\to\infty} \inf x_{n}\leq \lim_{k\to\infty} \inf x_{n_{k}} \leq \lim_{k\to\infty} \sup x_{n-{k}} \leq \lim_{n\to\infty} \sup x_{n}.$$ 
\end{prop}

\begin{prop}[Sub-Check]
A bounded sequence $\left\{x_{n}\right\}_{n=1}^{\infty}$ is convergent and converges to $x$ if and only if every convergent subsequence $\left\{x-{n_{k}}\right\}_{k=1}^{\infty}$ converges to x.
\end{prop}

With \textbf{Unbounded Sequence}, we can define:
\begin{defi}
    We say $\left\{x_[n]\right\}_{n=1}^{\infty}$ diverges to infinity if for every $K\in\mathbb{R}$, there exists an
    $M\in\mathbb{N}$ such that for all $n\geq M$, we have $x_{n}>K$. We write 
    $$\lim_{n\rightarrow \infty} x_{n}=\infty.$$
    Similarly, for $x<K$, we have 
    $$\lim_{n\rightarrow \infty} x_{n}=-\infty.$$
\end{defi}

\begin{prop}
    Suppose $\left\{x_{n}\right\}_{n=1}^{\infty}$ is an unbounded sequence. Then 
    \[
    \lim_{n\to\infty} x_{n} =
    \begin{cases}
    \infty & \text{if } \left\{x_{n}\right\}_{n=1}^{\infty} \text{ is increasing} \\
    -\infty & \text{if } \left\{x_{n}\right\}_{n=1}^{\infty} \text{ is decreasing}
    \end{cases}
    \]
\end{prop}

\begin{defi}
Let $\left\{x_{n}\right\}_{n=1}^{\infty}$ be an unbounded sequence. Define $\left\{a_{n}\right\}_{n=1}^{\infty}$ and $\left\{b_{n}\right\}_{n=1}^{\infty}$ as above.
Then $\left\{a_{n}\right\}_{n=1}^{\infty}$ is decreasing and $\left\{b_{n}\right\}_{n=1}^{\infty}$ is increasing. And $\lim_{n\to\infty} \sup x_{n}=\inf {a_{n}:n\in\mathbb{N}}$ and $\lim_{n\to\infty} \inf x_{n}=\sup {b_{n}:n\in\mathbb{N}}$.
\end{defi}

\subsection{Monotune Sequence}

\begin{defi}
A sequence $\left\{x_{n}\right\}_{n=1}^{\infty}$ is monotone increasing if ${x_{n}\leq x_{n+1}},\ for\ n=1,2,\cdots$.
A sequence $\left\{x_{n}\right\}_{n=1}^{\infty}$ is monotone decreasing if ${x_{n}\geq x_{n+1}},\ for\ n=1,2,\cdots$.
If a sequence is either monotone increasing or monotone decreasing, we say that the sequence is monotone.
\end{defi}

\begin{theorem}
    A monotonic sequence converges if, and only if, it is bounded.

    Furthermore, if $\left\{x_{n}\right\}_{n=1}^{\infty}$ is monotone increasing and bounded, then $\lim_{n\to \infty} x_{n}=\sup{x_{n}:n\in \mathbb{N}}$.
    If $\left\{x_{n}\right\}_{n=1}^{\infty}$ is monotone decreasing and bounded, then $\lim_{n\rightarrow \infty} x_{n}=\inf{x_{n}:n=1,2,\cdots}.$
\end{theorem}


\begin{proof}
Consider a monotune increasing sequence $\left\{x_{n}\right\}_{n=1}^{\infty}$, if it is bounded, we set $x:=\sup{x_{n}:n\in \mathbb{N}}.$. Let $\epsilon>0$ be arbitary. As $x$
is the supremum of the sequence, we have at least one $M$ satisfying that $x_{M}>x-\epsilon.$ As ${x_{n=1}}^{\infty}_{n=1}$ is monotue increasing, for $n\leq M$, $|x_{n}-x|\leq|x-x_{M}|\leq \epsilon.$ 
Hence $\left\{x_{n}\right\}_{n=1}^{\infty}$  converges to $x$. From the other side, if $\left\{x_{n}\right\}_{n=1}^{\infty}$ is monotone increasing and convergent, then it is bounded. Vise versa, we can prove the situation of monotune decreasing sequence.
\end{proof}  

\begin{prop}
    Let $S\subset R$ be a nonempty subset of $R$. Then there exist monotone sequence $\left\{x_{n}\right\}_{n=1}^{\infty}$ and ${y_{n}}_{n=1}^{\infty}$ such that $x_{n},y_{n}\in S$ and
    $$\sup S = \lim_{n\to \infty}x_{n}\ \ and \inf S = \lim_{n\to \infty}y_{n}.$$ 
\end{prop}




\section{Infinite Series}
Let ${a_{n}}$ be a sequence of real or comcplex
numbers, and form a new sequence ${s_{n}}$ as follows:
$$s_{n}=a_{1}+a_{2}+\cdots +a_{n}\ (n=1,2,\cdots)$$

\subsection{Basis}

\begin{defi}[Series]
The orderd pair of sequences ${a_{n}},{s_{n}}$ is 
called an infinite series. The number $s_{n}$ is called 
the \textit{nth partial sum} of the series. The series is said 
to \textit{converge} or to \textit{diverge} according 
as ${s_{n}}$ is convergent or divergent. The following 
symbols are used to denote series:
$$a_{1}+a_{2}+\cdots +a_{n}+\cdots,\ ,a_{1}+a_{2}+a_{3}+\cdots,\ ,\sum_{k=1}^{\infty}a_{k}.$$
\end{defi}

\begin{defi}[Patial Sum]
A series converges if the sequence ${s_{k}}_{k=1}^{\infty}$ defined by 
$$s_{k}:=\sum_{n=1}^{k}x_{n}=x_{1}+x_{2}+x_{3}+\cdots +x_{k}+\cdots$$
converges. The numbers $s_{k}$ are called the \textit{partial sums}.
\end{defi}

\textbf{Note.} The letter $k$ used in $\sum_{k=1}^{\infty}a_{k}$ 
is a \textbf{"dummy variable"} and may be replaced by 
any other letter.

If the sequence ${s_{n}}$ defined as previous converges to $s$, the 
number $s$ is called the \textit{sum} of the series, and we write
$$ s=\sum_{k=1}^{\infty}a_{k}$$.

\begin{corollary}
    Let $a=\sum a_{n}$ and $b=\sum b_{n}$ be convergent 
    series. Then, for every $\alpha$ and $\beta$, the series $\sum (\alpha a_{n}+\beta b_{n})$
    converges to the sum $(\alpha a+\beta b)$.
\end{corollary}
\begin{proof}
    $\sum_{k=1}^{n} (\alpha a_{k}+\beta b_{k})=\alpha \sum_{n=1}^{\infty}a_{n}+\beta \sum_{n=1}^{\infty}b_{n}.$
\end{proof}

\begin{corollary}
    Assume that $a_{n}\neq 0$ for each $n=1,2,\cdots$ Then $\sum a_{n}$
    converges if, and only if, the sequence of patial sums is bounded above.
\end{corollary}
\begin{proof}
    Let $s_{n}=a_{1}+a_{2}+\cdots+a_{n}$. Then we can apply the theorem.
\end{proof}

\begin{theorem}[Telescoping series]
    Let ${a_{n}}$ and ${b_{n}}$ be two sequences 
    such that $a_{n}=b_{n+1}-b_{n}$ for $n=1,2,\cdots$.
    Then $\sum a_{n}$ converges if, and only if, 
    $\lim_{n\rightarrow \infty}\sum b_{n}$ exists, in which 
    case we have $\sum_{n=1}^{\infty} a_{n}=\lim_{n\rightarrow \infty}b_{n}-b_{1}$.
\end{theorem}

\begin{theorem}[Cauchy condition for series]
    The series $\sum a_{n}$ converges if, and only if,
    for every $\epsilon>0$, there exists an integer $N$ such that
    $n>N$ implies 
    $$|a_{n+1}+a_{n+2}+\cdots+a_{n+p}|<\epsilon\ for\ each\ p=1,2,\cdots$$
\end{theorem}

Taking $p=1$ in the previous theorem, we find that $\lim_{n\rightarrow \infty}a_{n}=0$ is 
a necessary condition for the convergence of $\sum a_{n}$. That this condition is not sufficient
to ensure the convergence of $\sum a_{n}$ is shown as follows as we choose $a_{n}=\frac{1}{n}$:
$$a_{n+1}+\cdots+a_{n+p}=\frac{1}{2^{m}+1}+\cdots+\frac{1}{2^{m}+2^{m}}\geq\frac{2^{m}}{2^{m}+2^{m}}=\frac{1}{2},$$
and hence $\sum a_{n}$ diverges. This series is called the \textit{harmonic series}.

\begin{prop}
    Let $\sum_{n=1}^{\infty}x_{n}$ be a convergent series. Then the sequence 
    $\left\{x_{n}\right\}_{n=1}^{\infty}$ is convergent and $$\lim_{n\rightarrow \infty}x_{n} = 0.$$
\end{prop}

\begin{proof}
    
\end{proof}


% todo 

\begin{theorem}[Cauchy Condition for series]
    The series $\sum a_{n}$ converges if, and only if, for every $\epsilon>0$ there exists an integer $N$ such that $n>N$ implies
    $$|a_{n+1}+a_{n+2}+\cdots+a_{n+p}|<\epsilon\ for\ each\ p=1,2,\cdots.$$
\end{theorem}


\subsection{Inserting and Removing Parentheses}

\newpage
\section{Exercise}


\begin{exe}
Find : $\underset{n \to \infty}{\lim}\frac{n}{e^{n}}.$ 
\end{exe}
\begin{sol}
    Set continous function $f(x)=\frac{x}{e^{x}}$ and $\lim_{x\to \infty}f(x)=\underset{n \to \infty}{\lim}\frac{n}{e^{n}}$
Since $\underset{x \to \infty}{\lim}x=\infty=\underset{x \to \infty}{\lim}e^{x},$ we can use L'Hôpital's Rule.
\begin{align*}
    &\underset{x \to \infty}{\lim}\frac{x}{e^{x}}\\
    =& \underset{x \to \infty}{\lim}\frac{1}{e^{x}}\\
    =0
\end{align*}
Thus the limit is $0$.
\end{sol}

\begin{exe}
Find : $\underset{n \to \infty}{\lim}\frac{\sin{n}}{n}.$
\end{exe}

\begin{sol}
    Since $$-1\leq \sin{n}\leq 1,$$ we have $$-\frac{1}{n}\leq \frac{\sin{n}}{n}\leq \frac{1}{n}.$$
    As $n$ approaches infinity, the limit of the sequence $\left\{-\frac{1}{n}\right\}$ and $\left\{\frac{1}{n}\right\}$ is both $0$. Using the squeezing theorem, we have $$\underset{n \to \infty}{\lim}\frac{\sin{n}}{n}=0.$$
\end{sol}

\begin{exe}
    Find : $\underset{n \to \infty}{\lim}\frac{\ln(1+\frac{1}{n})}{n}.$
\end{exe}
\begin{sol}
\begin{align*}
    &\underset{n \to \infty}{\lim}\frac{\ln(1+\frac{1}{n})}{n}\\
    =& \frac{\underset{n \to \infty}{\lim} \ln(1+\frac{1}{n})}{\underset{n \to \infty}{\lim} n}\\
    =&\frac{0}{\infty}\\
    =&0	
\end{align*}
\end{sol}

\begin{exe}
    Find : $\underset{n \to \infty}{\lim}n\sin^{2}(\frac{1}{n}).$
\end{exe}

\begin{sol}
    \begin{align*}
        &\underset{n \to \infty}{\lim}n\sin^{2}(\frac{1}{n})\\
        =& \underset{n \to \infty}{\lim}\frac{\sin^{2}(\frac{1}{n})}{\frac{1}{n}}\ for\ x\in R^{+} \lim_{x\to \infty}\sin^{2}(\frac{1}{x})=0=\lim_{x\to \infty}\frac{1}{x}, using L'Hôpital's Rule \\
        =& \underset{x \to \infty}{\lim} 2\sin(\frac{1}{x})\cos(\frac{1}{x})\\
        =& 0
    \end{align*}
\end{sol}


\begin{exe}
    Find : $\underset{n \to \infty}{\lim}(n+\frac{1}{n})^{\frac{1}{n}}.$
\end{exe}

\begin{sol}
    \begin{align*}
        &\underset{n \to \infty}{\lim}(n+\frac{1}{n})^{\frac{1}{n}}\\
        =& \underset{n \to \infty}{\lim}e^{\frac{\ln(n+\frac{1}{n})}{n}}\  \\
        =& \exp{\underset{n \to \infty}{\lim}\frac{\ln(n+\frac{1}{n})}{n}}\ Obviously\ \underset{n \to \infty}{\lim}\frac{\ln(n+\frac{1}{n})}{n}=0 \\
        =& 1
    \end{align*}
\end{sol}


\begin{exe}
    Find : $\underset{n \to \infty}{\lim}\sqrt[n]{\ln n}.$
\end{exe}

\begin{sol}
Since $$1\leq \ln n\leq n,\ for\ n\leq 3$$
we have $$1\leq \sqrt[n]{\ln n}\leq \sqrt[n]{n},\ for\ n\leq 3.$$

As we have shown that $\lim_{n\to \infty}\sqrt[n]{n}=1$, then use the squeezing theorem . We have $$\underset{n \to \infty}{\lim}\sqrt[n]{\ln n}=1.$$

\end{sol}

\begin{exe}
    Prove that $a_{n}=\sum_{i=1}^{n}\frac{1}{i^{\alpha}}
    \begin{cases}
    is\ divevrgent\ for\ \alpha=1\\
    is\ convergent\ for\ \alpha>1
    \end{cases}.$
\end{exe}


\begin{sol}

For $\alpha\geq 2$ and all $\epsilon>0$ , there exists $N=[\frac{1}{\epsilon}]$, such that for $n\geq N$, 
\begin{align*}
    0<|a_{n+p}-a_{n}|=& \sum_{i=1}^{p}\frac{1}{(n+i)^{\alpha}}\\
    \leq& \sum_{i=1}^{p}\frac{1}{(n+i)^{2}}\\
    <& \sum_{i=1}^{p}\frac{1}{(n)(n+i)}\\
    =& \frac{1}{n}-\frac{1}{n+p}\\
    <& \frac{1}{n}\\
    <& \epsilon
\end{align*}

For $\alpha<1$, 
\begin{align*}
    a_{n+p}-a_{n}=& \sum_{i=1}^{p}\frac{1}{(n+i)^{\alpha}}\\
    >& \sum_{i=1}^{p}\frac{1}{n+i}\\
    >& \frac{p}{n+p}\\
    >& \frac{1}{2}
\end{align*}

There exists $\epsilon=\frac{1}{2}$, such that for all $N$, $\exists n_{0}=p\geq N$, such that
$$|a_{n+p}-a_{n}|>\frac{1}{2}=\epsilon\ for\ n\geq n_{0}.$$
\end{sol}

\begin{exe}
    Let $\left\{x_{n}\right\}_{n=1}^{\infty}$ be a sequence. Suppose there are two convergent subsequence $\left\{x_{n_{i}}\right\}_{i=1}^{\infty}$ and
    $\left\{x_{m_{i}}\right\}_{i=1}^{\infty}$. Suppose 
    $$\lim_{i\to \infty}x_{n_{i}}=a\ \ and\ \ \lim_{i\to \infty}x_{m_{i}}=b,$$
    where $a\neq b$. Prove that $\left\{x_{n}\right\}_{n=1}^{\infty}$ is not convergent.
\end{exe}

\begin{exe}[Homework]
\end{exe}

\begin{sol}[2.1.16]

    Suppose $\left\{x_{n}\right\}_{n=1}^{\infty}$ is convergent and converges to $L$. Then from the definition we know that for every $\epsilon$, there exists an $M$ that for all $n\geq M$,
    $$ |x_{n}-L|<\epsilon.$$
    
    For all $i\geq M$, we have $n_{i}\geq M$, since $\left\{x_{n_{i}}\right\}_{i=1}^{\infty}$ is a subsequence of $\left\{x_{n}\right\}_{n=1}^{\infty}$, we can have 
    $$|x_{n_{i}}-L|<\epsilon\ for\ i\geq M.$$
    And we know that $\left\{x_{n_{i}}\right\}_{i=1}^{\infty}$ converges to $a$, thus we have $a=L$.
    
From the other side, we know that the subsequence $\left\{x_{m_{i}}\right\}_{i=1}^{\infty}$ also converges to $L$, thus we have $b=L$.
Thus we have $a=b=L$, which contradicts the fact that $a\neq b$.
Hence $\left\{x_{n}\right\}_{n=1}^{\infty}$ is not convergent.
\end{sol}

\begin{sol}[2.1.17]

The sequence of all rational numbers.
\end{sol}


\begin{sol}[2.1.20]
We can know that ${y_{n}}_{n=1}^{\infty}$ is a subsequence of the sequence $\left\{x_{n}\right\}_{n=1}^{\infty}$. If $\left\{x_{n}\right\}_{n=1}^{\infty}$ is 
convergent, then ${y_{n}}_{n=1}^{\infty}$ is also convergent as the subsequence of $\left\{x_{n}\right\}_{n=1}^{\infty}$ and $\lim_{n\to \infty}x_{n}=\lim_{n\to \infty}y_{n}$.

If ${y_{n}}_{n=1}^{\infty}$ is convergent, 

\end{sol}

\begin{sol}[2.1.22]

Suppose $\left\{x_{2n}\right\}_{n=1}^{\infty}$, $\left\{x_{2n-1}\right\}_{n=1}^{\infty}$ and $\left\{x_{3n}\right\}_{n=1}^{\infty}$ converge to $L_{1}$,$L_{2}$ and $L_{3}$ respectively. We know that $\left\{x_{3n}\right\}_{n=1}^{\infty}$ 
is a subsequence of $\left\{x_{2n-1}\right\}_{n=1}^{\infty}$, thus we can have that $L_{1}=L_{3}$. We know that ${x_{6n}}$ is the subsequence of $\left\{x_{2n}\right\}_{n=1}^{\infty}$ and $\left\{x_{3n}\right\}_{n=1}^{\infty}$, and we assume that 
$\lim_{n\to \infty}x_{6n}=L'$, then we have $L'=L_{2}=L_{3}$. Hence $L_{1}=L_{2}=L_{3}$ and we set it as $L$.

Consider the convergence of $\left\{x_{2n}\right\}_{n=1}^{\infty}$ and $\left\{x_{2n-1}\right\}_{n=1}^{\infty}$, we can have that for 
all $\epsilon>0$, there exists $N_{1}$ and $N_{2}$ such that for all $n\geq N_{1}$ and $n\geq N_{2}$,
$$|x_{2n}-L|<\epsilon\ and\ |x_{2n-1}-L|<\epsilon.$$
We can have that $N=max\{N_{1},N_{2}\}$. Then we can have that for all $n\geq N$,
$$|x_{n}-L|<\epsilon.$$
then we have that $\left\{x_{n}\right\}_{n=1}^{\infty}$ converges to $L$.
\end{sol}

\begin{sol}[2.1.23]

Suppose $\left\{x_{n}\right\}_{n=1}^{\infty}$ is a monotune increasing sequence and its subsequence $\left\{x_{n_{i}}\right\}_{i=1}^{\infty}$ converges to $L$.
It's easy to prove that $n_{k}\geq k$ for all $k\geq 1$. Then we have that for all $k\geq 1$, $x_{k}\leq x_{n_{k}}$, for ${x_{n_{k}}}_{k=1}^{\infty}$ is bounded especially upper-bounded, we can 
know that ${x_{k}}_{k=1}^{\infty}$ is bounded above. From the theorem we know that ${x_{k}}_{k=1}^{\infty}$ converges to $L$.
\end{sol}

\begin{sol}[2.2.6]

    $\lim_{n\to\infty}z_{n}=0\ and\ \lim_{n\to\infty}w_{n}=\infty$. No, because $\lim_{n\to\infty}x_{n}=0$ and $\lim_{n\to\infty}y_{n}=0$.
\end{sol}

\begin{sol}[2.2.7]

    $x_{n} = (-1)^{n} (1-e^{-n})$
\end{sol}

\begin{sol}[2.2.8]
    Set $f(x)=\frac{n^{x}}{2^{x}}$, and we have $\lim_{x\to\infty}f(x)=\lim_{n\to\infty}$.
    Then use l'Hôpital's Rule twice and we can  $\lim_{x\to\infty}f(x)=0$.And $\lim_{n\to\infty}=0$.
\end{sol}

\begin{sol}[2.2.9]

    For any $\epsilon>0$, there exists an $M$ such that for all $n\geq M$, we have 
    $$0<|\frac{|x_{n+1}|-x}{x_{n}-x}-L|<\epsilon.$$
    Thus \begin{align*}
        -\epsilon<& \frac{|x_{n+1}-x|}{|x_{n}-x|}-L<\epsilon\\
        -\epsilon+L<& \frac{|x_{n+1}-x|}{|x_{n}-x|}<L+\epsilon\\
        0<& \frac{|x_{n+1}-x|}{|x_{n}-x|}<L+\epsilon\\
    \end{align*}
    since $\epsilon$ is arbitrary small and $L<1$, we can choose that $L+\epsilon<1$, hence we apply multiple from the m-th term to the n-th term, 
    \begin{align*}
        0<& \frac{|x_{n}-x|}{|x_{M}-x|}<(L+\epsilon)^{n}<1\\
        0<& |x_{n}-x|<|x_{M}-x|*{(L+\epsilon)^{n}}<1\\
        0<& \lim_{n\to\infty}|x_{n}-x|<\lim_{n\to\infty}|x_{M}-x|*{(L+\epsilon)^{n}}\\
    \end{align*}
    since $|x_{M}-x|$ is a finite number, then $\lim_{n\to\infty}|x_{M}-x|*{(L+\epsilon)^{n}}=0$ and we have 
    $$\lim_{n\to\infty}|x_{n}-x|=0.$$
    That is to say that $x_{n}=x$.
\end{sol}

\begin{sol}[2.2.10]
    
\end{sol}

\begin{sol}[2.2.14]

    Now we want to show it's monotune and bounded. Fisrtly, 
    \begin{align*}
        &x_{n+1}-x_{n}=x_{n}^{2}\geq 0\\
        \Rightarrow &x_{n} \geq x_{1}=c\\
    \end{align*} 

    If $c>0$, we have that $$\frac{x_{n+1}}{x_{n}}=x_{n}+1\geq 1+c >0.$$
    And it diverges.

    If $c<-1$, we have that $x_{n+1}-x_{n}=x_{n}^{2}\geq c^{2}>1$, it's obvious that it diverges.
    
    If $-1\leq c\leq 0$, we have that 
    \begin{align*}
    &x_{2}=x_{1}(x_{1}+1)\in[-1,0]\\
    \Rightarrow &x_{3}=x_{2}(x_{2}+1)\in[-1,0]\\
    \Rightarrow &x_{4}=x_{3}(x_{3}+1)\in[-1,0]\\
    \vdots\\
    \Rightarrow &x_{n}=x_{n-1}(x_{n-1}+1)\in[-1,0]\\
    \end{align*}

    Thus $x_{n}$ is bounded and monotone increasing, it converges. And we can assume it converges to $x$. Then
\begin{align*}
    &x=x^{2}+x\\
    \Rightarrow & x=0
\end{align*}
    
    For $c\in[-1,0]$, we have that $x_{n}$ converges to $0$.

Now if $\left\{x_{n}\right\}_{n=1}^{\infty}$ converges. Then we can know that it converges to $0$. Since $x_{n}$ is not decreasing, then $c\leq 0$ and $x_{n}\leq0$.
And $c$ must be not smaller than $-1$ since $x_{2}=c(c+1)$. So $c\in[-1,0]$.
\end{sol}

\begin{sol}[2.2.15]
    \begin{align*}
        & \lim_{n\to\infty}(n^2+1)^{\frac{1}{n}}\\
        =& \lim_{n\to\infty}e^{\frac{\ln(n^2+1)}{n}}\\
        =& e^{\lim_{n\to\infty}\frac{\ln(n^2+1)}{n}}\\
        =& e^{\lim_{x\to\infty}\frac{\ln(x^2+1)}{x}}\\
        =& e^{\lim_{x\to\infty}\frac{2x}{1+x^2}}\\
        =& e^{\lim_{x\to\infty}\frac{2}{2x}}\\
        =& e^{0}\\
        =& 1
    \end{align*}
\end{sol}

\begin{sol}[2.1.16]

    \begin{align*}
        & \frac{C^{n+1}}{(n+1)!}/\frac{C^{n}}{n!}\\
        =& \frac{C}{n+1}\\
        \lim_{n\to\infty}\frac{C^{n+1}}{(n+1)!}/\frac{C^{n}}{n!}\\
        =& \lim_{n\to\infty}\frac{C}{n+1}\\
        =& 0
    \end{align*}

    Then $\lim_{n\to\infty}\frac{C^{n}}{n!}=0$, which means for large n, $n!$ is larger than $C^{n}$ and $(n!)^{\frac{1}{n}}>C$. Since $C$ is arbitary, and $(n!)^{\frac{1}{n}}$ 
    doesn't have an upper bound. That is it diverges.
\end{sol}


\begin{sol}[2.3.7,2.3.8]

Since $\left\{x_{n}\right\}_{n=1}^{\infty}$ and $\left\{y_{n}\right\}_{n=1}^{\infty}$ are both bounded sequences then $|x_{n}|<M_{x}$ and $|y_{n}|<M_{y}$. Thus 
$$|x_{n}+y_{n}|\leq |x_{n}|+|y_{n}| < M_{x}+M_{y},$$
and it is bounded.

Now we find a convergent subsequence of $\left\{x_{n}+y_{n}\right\}_{n=1}^{\infty}$ and denote it as $\left\{x_{n_{m}}+y_{n_{m}}\right\}_{m=1}^{\infty}$, and we can have that
hence $$\lim_{n\to\infty}x_{n_{m}}+y_{n_{m}}=L.$$
And we find a convergent subsequence of $\left\{x_{n_{m}}\right\}_{m=1}^{\infty}$ and denote it as $\left\{x_{n_{m_{i}}}\right\}_{i=1}^{\infty}$, and we can have that
hence $$\lim_{i\to\infty}x_{n_{m_{i}}}=x\ and\ \lim_{i\to\infty}x_{n_{m_{i}}}+y_{n_{m_{i}}}=L\ for\ the\ corresponding\ subsequence\ y_{n_{m_{i}}}.$$
and we can have that $$\lim_{i\to\infty}y_{i} = L-x.$$

Since $\lim_{n\to\infty}\inf x_{n}$ is an inferior bound of $\left\{x_{n}\right\}_{n=1}^{\infty}$, we can have that 
$$\lim_{n\to\infty}\inf x_{n} \geq x,$$
also 
$$\lim_{n\to\infty}\inf y_{n} \geq L-x.$$
and we have that $$\lim_{n\to\infty}\inf x_{n}+\lim_{n\to\infty}\inf y_{n} \leq \lim_{n\to\infty}\inf x_{n}+y_{n}.$$
The same is for $\lim_{n\to\infty}\sup x_{n}+\lim_{n\to\infty}\sup y_{n}\geq\lim_{n\to\infty}\inf x_{n}+y_{n}$.

\end{sol}

\begin{sol}[2.3.9]


    Let $S \subset \mathbb{R}$ be a bounded and infinite set. We need to prove that $S$ has at least one cluster point.

    Since $S$ is bounded, there exist real numbers $a$ and $b$ such that $S \subset [a, b]$.
    
    Now, construct a sequence of nested intervals as follows:
    
    \begin{enumerate}
        \item Divide $[a, b]$ into two subintervals: $[a, \frac{a+b}{2}]$ and $[\frac{a+b}{2}, b]$.
        \item Since $S$ is infinite, at least one of these subintervals contains infinitely many points of $S$. Choose one such subinterval and denote it by $[a_1, b_1]$.
        \item Repeat this process: divide $[a_n, b_n]$ into two subintervals and select one that contains infinitely many points of $S$, denoting it by $[a_{n+1}, b_{n+1}]$.
    \end{enumerate}
    
    By induction, we obtain a nested sequence of intervals $[a_n, b_n]$, where each $[a_{n+1}, b_{n+1}]$ is a subset of $[a_n, b_n]$, and $b_n - a_n = \frac{b-a}{2^n}$.
    
    According to the Nested Interval Theorem, the intersection of all these intervals contains at least one point $L$:
    
    \[ L = \bigcap_{n=1}^{\infty} [a_n, b_n] \]
    
    Now, we need to show that $L$ is a cluster point of $S$.
    
    Take any $\epsilon > 0$. Since $b_n - a_n = \frac{b-a}{2^n}$, there exists an $N$ such that for all $n \geq N$,
    
    \[ b_n - a_n < \epsilon \]
    
    For such $n$, since $[a_n, b_n]$ contains infinitely many points of $S$, there exists at least one point $x \in S$ with $x \neq L$ within $(L - \epsilon, L + \epsilon)$.
    
Therefore, $L$ is a cluster point of $S$.
\end{sol}

\begin{sol}[2.3.11]
\begin{enumerate}[label=\alph*]
\item Now we assume that $\left\{x_{n}\right\}_{n=1}^{\infty}$ is a unbounded sequence, then we can find a unbounded sequence $\left\{x_{n_{m}}\right\}_{m=1}^{\infty}$ and it can't have a convergent subsequence, contradicting to the fact. Thus the sequence is bounded.
\item If the sequence doesn't converge to $x$, then we can construct a subsequece $\left\{x_{n_{k}}\right\}_{k=1}^{\infty}$ such that for $\epsilon>0$, $|x_{n_{k}}-x|>\epsilon$ for all $k$. Obviously, it doesn't have a subsequence converges to $x$, contradicting to the fact. Thus the sequence converges to $x$.
\end{enumerate}

\end{sol}

\begin{sol}[2.3.12]
\begin{enumerate}[label=\alph*]
\item Since the sequence is bounded, we can find a upper bound $s$ of it, then for all $n\in\mathbb{N}$, $x_{n}<s$.
\item From the definition, $\lim_{n\to\infty}sup x_{n}=s$.
\item Yeah. It's right.
\end{enumerate}
\end{sol}


\begin{sol}[2.3.15]
\begin{enumerate}[label=\alph*]
\item We know that for any $\epsilon>0$, there exists an $N$ such that for all $n\geq N$, $|\frac{|x_{n+1}|}{|x_{n}|-L}<\epsilon$.
Thus we have that $|x_{n+1}|<(\epsilon+L)|x_{n}|$. Choosing $r=\epsilon+L<1$, we have that $|x_{n}|<r^{n-N-1}|x_{N}|$, thus $\left\{x_{n}\right\}_{n=1}^{\infty}$ converges to 0.
\item We can use the same procedure as before, and get $|x_{n+1}|>r|x_{n}|$ where $r=\epsilon+L>1$, thus $|x_{n}|>r^{n-N-1}|x_{N}|$ and as $n$ tends to $\infty$ $|x_{n}|$ tends to infinity and the sequence is not bounded.
\end{enumerate}    
\end{sol}


\begin{sol}[2.3.18]

Recalling the definition the sup and inf, we know that for any $\epsilon>0$, there exists an $N$ such that for all $n\geq N$, $\inf x_{n}> I-\epsilon$ and $\sup x_{n}< S+\epsilon$, \textit{(strictly I should verify deffrent $N_{1},N_{2}$, but for simplicity and not sacrificing its understanding, i use the neglecting the process of choosing $N=max(N_{1},N_{2})$)}where $S$ and $I$ is the finite limit.
Thus we have $$ I-\epsilon<\inf x_{n}\leq x_{n}\leq \sup x_{n}\leq S+\epsilon.$$
We denote it that the min and max of the sequence with $n\leq N$ is $x_{min},x_{max}$. Then for $\epsilon>0$, 
$$
\begin{cases}
    x_{min}\leq x\leq x_{max},\ for\ n\leq N\\
    I-\epsilon<x_{n}\leq S+\epsilon
\end{cases}
$$
And we conclude that they are convergent.
\end{sol}


\begin{sol}[2.3.19]

    Now we prove the case of $\sup$ first, and the case of $\inf$ is the same procedure.

    From the definition, for any $\delta>0$, there exists an $N_{1}$ such that for all $n\geq N_{1}$, $x_{n}<\sup x_{n}+\delta$. From the other side, we know that for any $\sigma>0$, there exists an $N_{2}$ such that for all $n\geq N_{2}$, $\sup x_{n}<\lim_{n\to\infty}\sup x_{n}+\sigma$. Then we have that $N=max(N_{1},N_{2})$.
And for $n\geq N$, we have that
$$x_{n}<\sup x_{n}+\delta<\lim_{n\to\infty}\sup x_{n}+\delta+\sigma.$$
And if we choose $\epsilon=\sigma+\delta$, we have that for all $n\geq N$, $x_{n}<\lim_{n\to\infty}\sup x_{n}+\epsilon$, that is $x_{n}-\lim_{n\to\infty}\sup x_{n}<\epsilon$.

Vise versa, the inequality from the other side can be proved by the same procedure.

\end{sol}

\begin{sol}[2.4.2]
We choose $\epsilon>0$, for any $n\ and\ k\geq N=[\log_{C}(\frac{\epsilon(1-C)}{|x_{2}-x_{1}|})]+1$,
    \begin{align*}
        |x_{n}-x_{n-1}|<&C^{n-1}|x_{2}-x_{1}|\\
        \cdots\\
        |x_{k+1}-x_{k}|<&C^{k}|x_{2}-x_{1}|\\
        since\ |x_{n}-x_{k}|<|x_{n}-x_{n-1}|+&\cdots+|x_{k+1}-x_{k}|,\ we\ have\\
        |x_{n}-x_{k}|<&\frac{C^{k+1}-C^{n}}{1-C}|x_{2}-x_{1}|\\
        <&\frac{C^{k+1}}{1-C}|x_{2}-x_{1}|\\
        <&\frac{C^{N}}{1-C}|x_{2}-x_{1}|\\
        =&\frac{C^{[\log_{C}(\frac{\epsilon(1-C)}{|x_{2}-x_{1}|})]+1}}{1-C}|x_{2}-x_{1}|\\
        <&\frac{C^{\log_{C}(\frac{\epsilon(1-C)}{|x_{2}-x_{1}|})}}{1-C}|x_{2}-x_{1}|\\
        <&\epsilon\\
    \end{align*}
Thus the sequence is Cauchy.
\end{sol}


\begin{sol}[2.4.4]
Easy, set the $M$ in $x_{n}$ to be $N$ in $y_{n}$, then you can prove it.
\end{sol}

\begin{sol}[2.4.5]
Set the limit is $L$. Suppose $L>0$, then set $\epsilon=\frac{L}{2}$, there exists an $N$ such that for all $n\geq N$, $|x_{n}-L|<\epsilon$, since there exists an $k\geq N$ that 
$x_{k}<0$, thus $|x_{k}-L|>L$, which contradicts to the fact that $|x_{k}-L|<\epsilon=\frac{L}{2}$. Then $L\leq0$. Vise versa, we can prove that $L\geq 0$ and $L=0$. The sequence converges to 0.

\end{sol}


\begin{sol}[2.4.6]
$k\geq\ n\geq N$, $\frac{n}{k^{2}}\leq\frac{1}{k}\leq\frac{1}{N}$, choose $N=[\frac{1}{\epsilon}]+1$ and you can prove it.
\end{sol}

\begin{sol}[2.4.7]

    For every $\epsilon>0$, there exists an $N$ such that for all $n,k\geq N$, $|x_{n}-x_{k}|<\epsilon$. Since there are infinitely many $x_{n}=c$, there exists an $k$ 
    such that $k\geq N$ and $x_{k}=c$. Thus $|x_{n}-c|<\epsilon$, which shows that $\lim_{n\to\infty} x_{n}=c$. 
    
\end{sol}

\begin{sol}[2.4.8]
    $a_{n}=e^{-\frac{n}{520}}$
\end{sol}









\end{document}