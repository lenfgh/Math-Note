\documentclass{article}
\usepackage[left=2cm,right=2cm,top=2cm,bottom=2cm]{geometry}
\usepackage{amsmath}
\usepackage{hyperref}
\usepackage{amssymb}
\usepackage{enumitem}
\usepackage{setspace}
\usepackage{amsthm}
\usepackage{thmtools}

% \declaretheorem[
%   name=Solution,
%   style=break
% ]{sol}

\newtheorem{theorem}{Theorem}[section]
\newtheorem{lemma}{Lemma}[theorem]
\newtheorem{corollary}{Corollary}[theorem]
\newtheorem{proposition}{Proposition}[section]



\theoremstyle{definition} 
\newtheorem{defi}{Definition}[section]
\newtheorem{exa}{Example}[defi]
\newtheorem{exe}{Exercise}[section]
\newtheorem{sol}{Solution}[exe]
\usepackage{lipsum}
\usepackage{titlesec}
\titlespacing*{\section}{0pt}{\baselineskip}{\baselineskip}
\newtheorem{pro}{Properties}[section]

\linespread{1.5}


\title{Real Numbers}
\author{Len Fu}
\date{11.29.2024}

\begin{document}

\maketitle

\begin{abstract}
                                     



\end{abstract}

\tableofcontents

\newpage


\section{Basic Properties}





\section{The Set of Real Numbers}


\subsection{}

\subsection{Archimedean Property}

\begin{theorem}
\begin{enumerate}
    \item (Archimedean Property) If $x,y\in \mathbb{R}$ and $x>0$, then there exists an $n\in \mathbb{N}$ such that $$nx>y.$$
    \item ($\mathbb{Q}$ is dense in $\mathbb{R}$) If $x,y\in \mathbb{R}$ and $x<y$, then there exists an $r\in \mathbb{Q}$ such that $$x<r<y.$$
\end{enumerate}
\end{theorem}

\begin{proof}
    Consider (i), for every real number $t:=\frac{y}{x}$
    
    Consider (ii), first assume $x\leq 0$, and $y-x>0$, then there exists an $n\in \mathbb{N}$ such that $n(y-x)>1$, and $y-x>\frac{1}{n}$.
    And there has a least integer $m>nx$, divide through by $n$ we get $x<\frac{m}{n}$.
    
    If $m>1$, then $m-1\in \mathbb{N}$ and $m-1\leq nx$. If $m=1$, $m-1=0\leq nx$. That is to say $nx\geq m-1.$

    Then $y>x+\frac{1}{n}\geq \frac{m}{n} >x$, that is $\mathbb{Q}$ is dense in $\mathbb{R}$.
\end{proof}

\subsection{Inf and Sup}

\begin{proposition}
Let $A,B\subset \mathbb{R}$ be nonempty sets such that $x\geq y$ whenever $x\in A$ and $y\in B$. Then $A$ is bounded above, $B$ is bounded below, and $\sup A\geq \inf B$. 
\end{proposition}

\begin{proof}




\end{proof}

\begin{proposition}
\end{proposition}

\begin{defi}Let $A\subset \mathbb{R}$ be a set.
\begin{enumerate}
\item If $A$ is empty, then $\sup A:=-\infty.$
\item If $A$ is empty, then $\inf A:=\infty.$
\item If $A$ is not bounded above, then $\sup A:=\infty.$
\item If $A$ is not bounded below, then $\inf A:=-\infty.$ 
\end{enumerate}

And $\mathbb{R}^{*}=\mathbb{R}\bigcup {\infty,-\infty}$ is defined as \textbf{the set of Extended Real Numbers}
\end{defi}

But we must leave $\infty-\infty,0\cdot\pm\infty,\ and\ \frac{\pm \infty}{\pm \infty}$ as undefined.


\subsection{Absolute Value and Bounded Functions}
\begin{proposition}[Triangle Inquality]
    Let $x,y\in \mathbb{R}$ and $x>0$, then $|x+y|\leq |x|+|y|.$
\end{proposition}

\begin{corollary}
    Let $x,y\in \mathbb{R}$.
    (i) (reverse triangle inequality) $||x|-|y||\leq |x-y|.$
    (ii) $|x-y|\leq |x|+|y|.$
\end{corollary}

\begin{defi}[Bounded Functions]
Suppose $f:D\rightarrow \mathbb{R}$ is a function. We say $f$ is \textbf{bounded} if there exists a constant $M\in \mathbb{R}$ such that $|f(x)|\leq M$ whenever $x\in D.$
\end{defi}

\begin{proposition}
If $f:D\rightarrow \mathbb{R}$ and $g:D\rightarrow \mathbb{R}$ are bounded functions and $f(x)\leq g(x)\ for\ all\ x\in D$
then
$$\sup_{x\in D}f(x)\leq \sup_{x\in D}g(x)\ and\ \inf_{x\in D}f(x)\leq\sup_{x\in D}g(x).$$
\end{proposition}

\subsection{Intervals and the size of $\mathbb{R}$}

\begin{proposition}
A set $I\subset \mathbb{R}$ is an interval if and only if $I$ contains at least 2 points and for all $a,c\in I$ and $a<b<c$, we have $b\in I$.
\end{proposition}

\subsection{Decimal Representation of the Reals}
We represent rational numbers with positive integer $M,K$ and digts $d_{K}d_{K-1}\cdots d_{1}d_{0}d_{-1}\cdots d_{-M+1}d_{M}$ such that 
$$x=d_{K}10^{K}+d_{K-1}10^{K-1}+\cdots +d_{1}10^{1}+d_{0}10^{0}+d_{-1}10^{-1}+\cdots +d_{-M+1}10^{-M+1}+d_{M}10^{-M}$$
and call $D_{n}$ the \textbf{truncation of $x$ to $n$ decimal digits}.

However for irrarional numbers, we can not represent them in this way. And for some infinite curcilation, we can not represent them in this way either.

For every real number $x\in(0,1]$, we define 
$$x=\sup_{n\in \mathbb{N}}D_{n}=\sup_{n\in \mathbb{N}}\left( \frac{d_{1}}{10}+ \frac{d_{2}}{10^{2}}+\dots +\frac{d_{n}}{10^{n}}\right).$$

\begin{proposition}
(i) Every infinite sequece of digts $0.d_{1}d_{2}\cdots$ represents a unique real number $x\in(0,1]$, and 
$$D_{n}\leq x\leq D_{n}+\frac{1}{10^{n}}\ for\ all\ n\in \mathbb{N}.$$
(ii) For every real number $x\in(0,1]$, there exists an infinite sequence of digts $0.d_{1}d_{2}\cdots$ that represents $x$. 
There exists a unique representation such that 
$$D_{n}< x\leq D_{n}+\frac{1}{10^{n}}\ for\ all\ n\in \mathbb{N}.$$
\end{proposition}


\begin{proposition}
    If $x\in (0,1]$ is a rational number and $x=0.d_{1}d_{2}\cdots$, then the decimal digits eventually start repeating. That is, there are 
    positive integers $N$ and $P$, such that for all $n\geq N$, $d_{n}=d_{n+P}$.
\end{proposition}


\newpage
\section{Exercise}

\begin{sol}[1.1.2]
    Since $A$ is a subset of ordered set $S$, we suppose the number of its elements is $n$ and denote it as $A_{n}$. Using the induction, we have:

    \noindent\textbf{Base Case:} If $n=1$, the only element is both the infimum and supremum of $A_{1}$, and $A_{1}$ is bounded.

    \noindent\textbf{Induction Step:} Assume the hypothesis holds for $n=k$, then we can find a smallest and a largest element $a_{k}$ and $b_{k}$ in $A$, 
    then we insert an element $x$ of $S$ into $A_{k}$ and regard it as $A_{k+1}$. Since $S$ is ordered, either $x>a_{k}$ or $a_{k}>x$ and there must have a smallest element in $A_{k+1}$, furthermore it is the infimum of $A_{k+1}$ and in $A_{k+1}$.
    Similarly, we can find the largest element as the supremum of $A_{k+1}$. And obviously $A_{k+1}$ is bounded.

    \noindent\textbf{Conclusion:} By the principle of induction, we have shown that for any $n\in \mathbb{N}$, $A_{n}$ is bounded. That is for every nonempty subset of ordered set, it is bounded with infimum and supremum within it.
\end{sol}

\begin{sol}[1.1.3]
Using proposition(ii):
    \begin{align*}
        x+y>0+0=0\ y-x>&0\\
        (y-x)(y+x)>&0\\
        y^{2}-x^{2}&>0\\
        y^{2}>&x^{2}
    \end{align*} 
\end{sol}


\begin{sol} 1.1.4\\
    $A$ is an ordered subset of ordered subset $B$, since all infs and sups exist, from the definition we know that:
    $$there\ exists\ an\ \sup A\ \in B,\ for\ all\ x\in A,\ x\leq \sup A.$$
    And 
    $$there\ exists\ an\ \sup B\ \in S,\ for\ all\ x\in B,\ x\leq \sup B.$$
    Since $\sup A$ is in $B$, then $\sup A\leq \sup B.$ Vise versa, $\inf B\leq\inf A.$

    And for a nonempty set with inf and sup, it obeys that $\inf\leq\sup$, thus $\inf A\leq\sup A$.
    
    Above all, we have proved that 
    $$\inf B\leq\inf A\leq\sup A\leq\sup B.$$
\end{sol}

\begin{sol}[1.1.5]
    We assume the supremum exists and denote the supremum of $A$ as $\sup A$. From the defition of supremum, since $b\in A$, we get that $b\leq \sup A$.
    From another side, we know that $b$ is an upper bound of $A$, thus $b\geq \sup A$. Obviously $b=\sup A.$
\end{sol}


\begin{sol} 1.2.3\\
    To prove $(iii)$, we suppose that $b$ is an upper bound of $A$, that is, $y\leq b$ for all $y\in A$. For $x>0$ we have 
    $xy\leq xb$ for all $y\in A$, and so $xb$ is an upper bound of $xA$. In particular, $b$ is sup of $A$.
    We have $\sup xA\leq x\sup A.$

    To prove the inverse inequality, suppose $c$ is a upper bound of $xA$, thus $xy\leq c$ for all 
    $y\in A$, and we have $y\leq \frac{c}{x}$ which reveals that $\frac{c}{x}$ is an upper bound of $A$. In particular, $c$ is the sup of $xA$, we have 
    $\sup A\leq \frac{\sup xA}{x}$.And we have $\sup xA=x\sup A$. Vise versa, it remains for $inf$ as $(iv)$.
    
    To prove $(v)$, we suppose that $b$ is an lower bound of $A$, that is, $y\geq b$ for all $y\in A$. For $x<0$ we have
    $xy\leq bx$ for all$ y\in A,$ and $bx$ is an upper bound of $xA$. In particular, $b$ is inf of $A$.We have $\sup xA\leq x\inf A.$
    
    To prove the inverse inequality, suppose $c$ is a upper bound of $xA$, thus $xy\leq c$ for all $y\in A$, and we have $y\geq \frac{c}{x}$ which reveals that $\frac{c}{x}$ is an lower bound of $A$. In particular, $c$ is the sup of $xA$.
    We have $\sup xA\geq x\inf A.$ And we have $\sup xA=x\inf A$ Vise versa, it remains for $sup$ as $(vi)$.
\end{sol}

\begin{sol}[1.2.5]
    Now we assume that $\sqrt{3}$ is rational and denote it as $\frac{p}{q}$ where $p,q$ are irreducible.
    Then we have $p^{2}=3q^{2}$, we can see that $p=3k$ for some $k\in N^{*}$, then $q^{2}=3k^{2}$. We conclude that 
    both $p$ and $q$ are multiple of 3, contradicting to the assumption. So the assumption fails, $\sqrt{3}$ is irrarional.
\end{sol}


\begin{sol}[1.2.8]
    For every pair of $x,y\in \mathbb{R}$, we have that $\frac{x}{\sqrt{2}},\frac{y}{\sqrt{2}}\in \mathbb{R}.$
    Since $\mathbb{Q}$ is dense in $\mathbb{R}$, we have that $\frac{x}{\sqrt{2}}<r<\frac{y}{\sqrt{2}}\in \mathbb{R}$ for some $r\in\mathbb{Q}$.
    Then we have that $x<\sqrt{2}r<y$, which implies that there exists an irrarional number $r^{*}$ such that $x<r^{*}<y.$
\end{sol}

\begin{sol} 1.2.9\\
    We set $p$ and $q$ is an upper bound of $A$ and $B$ correspondingly, for all $a\in A,b\in B$. Now we set $c=a+b\in C$, and we have 
    $$c=a+b\leq p+b\leq p+q$$
    we can see that $C$ is upper bounded. Then $p+q$ is an upper bound of $C$ and in particular, $p,q$ are sup of $A$ and $B$ respectively. Then we have $\sup C\leq\sup A+\sup B.$

    To prove the inverse inequality, as we have known that $C$ is upper bounded, we set $c$ as an upper bound of $C$. and for all $a\in A,b\in B$, we have $a+b\leq c$, then $a\leq c-b$ for
    all $a\in A$ showing that $c-b$ is an upper bound of $A$ and in particular, $c$ is sup of $C$. Then $\sup C-b$ is an upper bound of $A$ and we have $\sup A\leq \sup C-b$, or equally, 
    $b\leq \sup C-\sup A$. Follow the same procedure, we have $\sup B\leq \sup C-\sup A$ that is $\sup A+\sup B\leq \sup C$. And we see that $\sup A+\sup B=\sup C$. Vise versa, it remains the same 
    as it changes from sup to inf.
\end{sol}


\begin{sol}[1.2.10]
    Emmmmm, I don't think it differs in a large extent from the thinking chain of exercise[1.2.3] and exercise[1.2.9]. So let me skip this exercise.
\end{sol}

\begin{sol} 1.2.11\\
    To prove the statement, we first take the set $A=\left\{a\in \mathbb{R}|a^{n}<x\right\}.$ We need to show that $A$ is bounded above and has a supremum, which can be proved that it is the unique $r=x^{\frac{1}{n}}$ we want.
  
\noindent\textbf{Step1(Ensure the exsistence of supA):} For $x>1$, if $a>x$, we have $a^{n}>x^{n}$ contradicting to the assumption, thus $a<x$ which reveals that $A$ is upper bounded.
For $x<1$, then $a$ should be less than $1$, which reveals that $A$ is upper bounded. And whether $x$ is larger than 1 or not, $\frac{x}{2}^{n}<x^{n}$, thus $A$ is not empty. Thus there must exist the supremum.

\noindent\textbf{Step2(Show $r=x^{\frac{1}{n}}$):} Suppose the sup of $A$ is $r$.

Now we assume that $r^{n}<x$, and we first choose a number $0<h<1$. We can have
\begin{align*}
    &(r+h)^{n}-r^{n}\\
    =&h*Poly(r,h) \ (where\ Poly(r,h)=\sum a_{i}r^{i}h^{n-i-1}\ a_{i}>1)\\
    <&h*Poly(r,1)
\end{align*}
Then we set $h<\frac{x-r^{n}}{Poly(r,1)}$, we have
$$(r+h)^{n}<x.$$

That is to say, there exists a number $h>0$ such that $(r+h)^{n}<x$. And we know that $r+h\in A$ and $(r+h)>r$ 
contradicting to $r=\sup A$, thus $r^{n}\geq x$.

And now we assume that $r^{n}>x$, then we set $0<h<1$, and we have
\begin{align*}
    &r^{n}-(r-h)^{n}\\
    =&h*Poly(r,-h) \ (where\ Poly(r,-h)=\sum a_{i}r^{i}(-h)^{n-i-1}\ a_{i}>1)\\
    <&h*Poly(r,1)
\end{align*}
Then we set $h<\frac{r^{n}-x}{Poly(r,1)}$, we have
$$(r-h)^{n}>x.$$

That is to say there exists a number $h>0$ such that $(r-h)^{n}>x$. And we know that $r-h\notin A$ and there doesn't exisit an $x\in[r-h,r]$ satisfying 
$x\in A$ contradicting to $r=\sup A$(proposition 1.2.8 basic property of sup), thus $r^{n}\leq x$.

Ok then we have $r^{n}=x$~~. To prove its uniqueness, suppose that there are two numbers $r_{1},r_{2}$ satisfying, and we assume that $r_{1}<r_{2}$, and we can get 
$x<x$ as a consequence. Obviously it's wrong, thus $r_{1}=r_{2}$. And we ensure the uniqueness of $r$ by contradiction.

\end{sol}

\begin{sol} 1.2.13\\
Using principle of induction, we have:
\noindent\textbf{Base Case:} If $n=1$, the inequality is trivially satisfied.
\noindent\textbf{Induction Step:} Assume the inequality holds for $n=k$, then we can write
$$(1+x)^{k}-(1+kx)\geq 0.$$
Now we consider the inequality for $n=k+1$:
\begin{align*}
    &(1+x)^{k+1}-(1+(k+1)x)\\
    =&(1+x)^{k}(1+x)-1-kx-x\\
    \geq&(1+kx)(1+x)-1-kx-x\\
    =&kx^{2}\\
    \geq 0
\end{align*}
Obviously, the inequality holds for $n=k+1$ as well.
\noindent\textbf{Conclusion:} By the principle of induction, we have shown that for any $n\in \mathbb{N}$, the inequality is satisfied.

\end{sol}

\begin{sol} 1.2.15
\begin{enumerate}[label=(\alph*))]
\item  We set $A=\left\{x\in\mathbb{Q}|x<y\right\}$. First $y$ is an upper bound of $A$, we need to prove that $y$ is the sup of $A$. \\begin{singlespace}
    $A$ is upper bounded and nonempty, the sup is exsisting and we denote it as $r$. Assume that $r\neq y$, that is equally $r<y$. Since $\mathbb{Q}$ is dense in $\mathbb{R}$,
    there exsists an rational number $x$ such that $r<x<y$. Then we know that $x\in A$, then $x<r$, which contradicts to $r\neq y$. Thus $y=r$ is the sup of $A$.
\item We set $\inf A$ as $y$, from the definition of Dedekind cuts, we know that there is no largest element in $A$, that is for any $a\in A$, it must be $a<y$.
Thus $A\subset \left\{x\in\mathbb{Q}|x<y\right\}$.

Now we choose $b\in \left\{x\in\mathbb{Q}|x<y\right\}$, since $y$ is the sup of $A$, then for any $\epsilon>0$, there exists $a\in A$, satisfying $y-\epsilon<a<y$. We choose $y-x$ as $\epsilon$, then 
$x<a$ and we know that $x\in A$, that is $\left\{x\in\mathbb{Q}|x<y\right\}\subset A.$ 

And we have $$A=\left\{x\in\mathbb{Q}|x<y\right\},\ where\ y=\sup A.$$
\item $f:\mathbb{R}\rightarrow Dedekind\ Cuts$, $f(r)=\left\{x\in\mathbb{Q}|x<r\right\}\ r\in R.$
\end{enumerate}
\end{sol}


\begin{sol}[1.3.3]
    Skip.
\end{sol}

\begin{sol}[1.3.4]
    If $a$ is a lower bound of $f(D)$, then $a\leq f(x)\leq g(x)$, thus $a$ is also a lower bound of $g(D)$ and we choose the inf of $f(D)$.
    $$\inf_{y\in D} f(y)\leq g(x),\ for\ all\ x\in D$$
    and $\inf_{y\in D}f_{y}$ is a lower bound of $g(D)$ and less than the inf of $g(D)$:
    $$\inf_{x\in D}f(x)\leq \inf_{x\in D}g(x).$$
\end{sol}

\begin{sol} 1.3.5 
\begin{enumerate}[label=(\alph*)]
\item 
Since $f(x)\leq g(y)$ for all $x\in D$ and $y\in D$, then $g(y)$ is an upper bound of $f(D)$, $\sup_{x\in D}\leq g(y)$ for all $y\in D$. Then $\sup_{x\in D}$ is 
a lower bound of $g(D)$, and we get $\inf_{x\in D}\sup_{x\in D}\leq g(y).$
\item $D=[0,1],f(x)=x,g(x)=x+0.5.$
\end{enumerate}
\end{sol}

\begin{sol}[1.3.6]
    Now we rewrite the proposition's condition: If $f:D\rightarrow \mathbb{R^{*}}$ and $g:D\rightarrow \mathbb{R^{*}}$\dots

    Now the inf and sup is well defined even $f$ and $g$ are not bounded functions. And the proving procedure remains unchanged.
\end{sol}

\begin{sol} 1.3.7
\begin{enumerate}[label=(\alph*)]
\item For all $x\in D$, we have $f(x)\leq \sup_{x\in D} f(x)$ and $g(x)\leq \sup_{x\in D} g(x)$. Thus $f(x)+g(x)\leq \sup_{x\in D} f(x)+\sup_{x\in D} g(x)$ for all $x\in D$, and $\sup_{x\in D} f(x) +\sup_{x\in D} g(x)$ is an upper bound of $f(x)+g(x)$. And $$\sup_{x\in D}(f(x)+g(x))\leq\sup_{x\in D} f(x) +\sup_{x\in D} g(x).$$ 
Vise versa, it remains for the inf.
\item $\sin{x}$ and $\cos{x}$

\end{enumerate}
\end{sol}




\end{document}