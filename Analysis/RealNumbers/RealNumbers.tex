\documentclass{article}
\usepackage[left=2cm,right=2cm,top=2cm,bottom=2cm]{geometry}
\usepackage{amsmath}
\usepackage{hyperref}
\usepackage{amssymb}
\usepackage{enumitem}
\usepackage{setspace}
\usepackage{amsthm}


\newtheorem{theorem}{Theorem}[section]
\newtheorem{lemma}{Lemma}[theorem]
\newtheorem{corollary}{Corollary}[theorem]
\newtheorem{proposition}{Proposition}[section]

\theoremstyle{definition} 
\newtheorem{defi}{Definition}[section]
\newtheorem{exa}{Example}[defi]
\newtheorem{exe}{Exercise}[section]
\newtheorem{sol}{Solution}[exe]
\newtheorem{pro}{Properties}[section]

\title{Real Numbers}
\author{Len Fu}
\date{11.29.2024}

\begin{document}

\maketitle

\begin{abstract}
                                     



\end{abstract}

\tableofcontents

\newpage


\section{Basic Properties}





\section{The Set of Real Numbers}


\subsection{}

\subsection{Archimedean Property}

\begin{theorem}
\begin{enumerate}
    \item (Archimedean Property) If $x,y\in \mathbb{R}$ and $x>0$, then there exists an $n\in \mathbb{N}$ such that $$nx>y.$$
    \item ($\mathbb{Q}$ is dense in $\mathbb{R}$) If $x,y\in \mathbb{R}$ and $x<y$, then there exists an $r\in \mathbb{Q}$ such that $$x<r<y.$$
\end{enumerate}
\end{theorem}

\begin{proof}
    Consider (i), for every real number $t:=\frac{y}{x}$
    
    Consider (ii), first assume $x\leq 0$, and $y-x>0$, then there exists an $n\in \mathbb{N}$ such that $n(y-x)>1$, and $y-x>\frac{1}{n}$.
    And there has a least integer $m>nx$, divide through by $n$ we get $x<\frac{m}{n}$.
    
    If $m>1$, then $m-1\in \mathbb{N}$ and $m-1\leq nx$. If $m=1$, $m-1=0\leq nx$. That is to say $nx\geq m-1.$

    Then $y>x+\frac{1}{n}\geq \frac{m}{n} >x$, that is $\mathbb{Q}$ is dense in $\mathbb{R}$.
\end{proof}

\subsection{Inf and Sup}

\begin{proposition}
Let $A,B\subset \mathbb{R}$ be nonempty sets such that $x\geq y$ whenever $x\in A$ and $y\in B$. Then $A$ is bounded above, $B$ is bounded below, and $\sup A\geq \inf B$. 
\end{proposition}

\begin{proof}




\end{proof}

\begin{proposition}
\end{proposition}

\begin{defi}Let $A\subset \mathbb{R}$ be a set.
\begin{enumerate}
\item If $A$ is empty, then $\sup A:=-\infty.$
\item If $A$ is empty, then $\inf A:=\infty.$
\item If $A$ is not bounded above, then $\sup A:=\infty.$
\item If $A$ is not bounded below, then $\inf A:=-\infty.$ 
\end{enumerate}

And $\mathbb{R}^{*}=\mathbb{R}\bigcup {\infty,-\infty}$ is defined as \textbf{the set of Extended Real Numbers}
\end{defi}

But we must leave $\infty-\infty,0\cdot\pm\infty,\ and\ \frac{\pm \infty}{\pm \infty}$ as undefined.


\subsection{Absolute Value and Bounded Functions}
\begin{proposition}[Triangle Inquality]
    Let $x,y\in \mathbb{R}$ and $x>0$, then $|x+y|\leq |x|+|y|.$
\end{proposition}

\begin{corollary}
    Let $x,y\in \mathbb{R}$.
    (i) (reverse triangle inequality) $||x|-|y||\leq |x-y|.$
    (ii) $|x-y|\leq |x|+|y|.$
\end{corollary}

\begin{defi}[Bounded Functions]
Suppose $f:D\rightarrow \mathbb{R}$ is a function. We say $f$ is \textbf{bounded} if there exists a constant $M\in \mathbb{R}$ such that $|f(x)|\leq M$ whenever $x\in D.$
\end{defi}

\begin{proposition}
If $f:D\rightarrow \mathbb{R}$ and $g:D\rightarrow \mathbb{R}$ are bounded functions and $f(x)\leq g(x)\ for\ all\ x\in D$
then
$$\sup_{x\in D}f(x)\leq \sup_{x\in D}g(x)\ and\ \inf_{x\in D}f(x)\leq\sup_{x\in D}g(x).$$
\end{proposition}

\subsection{Intervals and the size of $\mathbb{R}$}

\begin{proposition}
A set $I\subset \mathbb{R}$ is an interval if and only if $I$ contains at least 2 points and for all $a,c\in I$ and $a<b<c$, we have $b\in I$.
\end{proposition}

\subsection{Decimal Representation of the Reals}
We represent rational numbers with positive integer $M,K$ and digts $d_{K}d_{K-1}\cdots d_{1}d_{0}d_{-1}\cdots d_{-M+1}d_{M}$ such that 
$$x=d_{K}10^{K}+d_{K-1}10^{K-1}+\cdots +d_{1}10^{1}+d_{0}10^{0}+d_{-1}10^{-1}+\cdots +d_{-M+1}10^{-M+1}+d_{M}10^{-M}$$
and call $D_{n}$ the \textbf{truncation of $x$ to $n$ decimal digits}.

However for irrarional numbers, we can not represent them in this way. And for some infinite curcilation, we can not represent them in this way either.

For every real number $x\in(0,1]$, we define 
$$x=\sup_{n\in \mathbb{N}}D_{n}=\sup_{n\in \mathbb{N}}\left( \frac{d_{1}}{10}+ \frac{d_{2}}{10^{2}}+\dots +\frac{d_{n}}{10^{n}}\right).$$

\begin{proposition}
(i) Every infinite sequece of digts $0.d_{1}d_{2}\cdots$ represents a unique real number $x\in(0,1]$, and 
$$D_{n}\leq x\leq D_{n}+\frac{1}{10^{n}}\ for\ all\ n\in \mathbb{N}.$$
(ii) For every real number $x\in(0,1]$, there exists an infinite sequence of digts $0.d_{1}d_{2}\cdots$ that represents $x$. 
There exists a unique representation such that 
$$D_{n}< x\leq D_{n}+\frac{1}{10^{n}}\ for\ all\ n\in \mathbb{N}.$$
\end{proposition}


\begin{proposition}
    If $x\in (0,1]$ is a rational number and $x=0.d_{1}d_{2}\cdots$, then the decimal digits eventually start repeating. That is, there are 
    positive integers $N$ and $P$, such that for all $n\geq N$, $d_{n}=d_{n+P}$.
\end{proposition}












\end{document}