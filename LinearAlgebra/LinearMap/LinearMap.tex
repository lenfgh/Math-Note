\documentclass{article}
\usepackage[left=2cm,right=2cm,top=2cm,bottom=2cm]{geometry}
\usepackage{amsmath} 
\usepackage{hyperref}
\usepackage{amsmath}
\usepackage{amssymb}
\usepackage{amsthm}

\newtheorem{theorem}{Theorem}[section]
\newtheorem{lemma}{Lemma}[theorem]

\theoremstyle{definition} 
\newtheorem{defi}{Definition}[section]
\newtheorem{example}{Example}[defi]
\newtheorem{exe}{Exercise}[section]
\newtheorem{sol}{Solution}[exe]
\newtheorem{pro}{Properties}[section]

\linespread{1.5} 


\title{Linear Mapping and Linear Transformation} 
\author{Len Fu} 
\date{11.19.24} 

\begin{document}


\maketitle

\begin{abstract}
This is the note of Linear Mapping and Linear Transformation, maded by Len Fu while his learning progress. 
The main content is from \textit{Linear Algebra Done Right}
\ and \textit{Linear Algebra\  Allenby}. It's also the notes from the classes of BIT.
\end{abstract}


\tableofcontents
\newpage

\section{Linear Mapping}
To prove a mapping is not linear mapping, you just need to find a counterexample.
\subsection{Mapping}
\begin{defi}[Mapping]
Suppose $X$ and $Y$ are two non-empty sets. A mapping $\sigma$
from $X$ to $Y$, denoted as $\sigma:X\rightarrow Y$, is a rule 
that assigns to each element $x\in X$ exactly one element $y$ 
in the set $Y$. The assignment $y=\sigma(x)$ is called the image of $a$
under the mapping $\sigma$, and $x$ is called the preimage. 
\end{defi}
\begin{pro}[Domain]
    Every element $x$ in the set $X$ must be mapped to some element in $Y$.
\end{pro}
\begin{pro}[Uniqueness]
    For each $x$ in $X$, there is a unique $y$ in $Y$ 
    such that $\sigma(x)=y$.
\end{pro}

The set $X$ is called the domain of the mapping $\sigma$, 
and the set $Y$ is called the codomain. The image of a set,
which is the set of all elements in $Y$ that are mapped 
to by elements in $X$, denoted by as $Im(\sigma) or \sigma(X)$.
, or $$\sigma(X)={y\in Y|\exists x\in X\ such\ that\ \sigma(x)=y}.$$

\begin{defi}[Injective Mapping]
    A mapping $\sigma$ is called an \textit{injective mapping} or an \textit{onto mapping}
    if for each $y$ in $Y$, there is a unique $x$ in $X$ such that $\sigma(x)=y$.
    Formally, for all $x_{1},x_{2}\in X$, if $\sigma(x_{1})=\sigma(x_{2})$, 
    then $x_{1}=x_{2}$.
\end{defi}

\begin{defi}[Surjective Mapping]
    A mapping $\sigma$ is called a \textit{surjective mapping} or a \textit{onto mapping}
    if for every $y$ in $Y$, there exists an $x$ in $X$ such that $\sigma(x)=y$.
\end{defi}

\begin{defi}[Bijective Mapping]
    A mapping $\sigma$ is called a \textit{bijective mapping} or a \textit{onto mapping}
    if it is both injective and surjective.
\end{defi}

\begin{defi}[Product of mappings]
    Set $\sigma$ as a mapping from $X$ to $Y$, and $\tau$ as a mapping from $Y$ to $Z$, 
    then we can define a new mapping $\tau\circ\sigma$ from $X$ to $Z$ by
    $$\tau\circ\sigma(x)=\tau(\sigma(x)),\ for\ all\ x\in X.$$
\end{defi}

\subsection{Linear Mapping}
\begin{defi}[Linear Mapping]
    Set the $V_{1}$ and $V_{2}$ as vector spaces on the field $F$. 
    If a mapping $\tau$ from $V_{1}$ to $V_{2}$ keeps the adding property and the scalar 
    multiplication property, then we say that $\tau$ is a \textit{linear mapping} or a
    \textit{linear transformation}.
    $$\sigma (\alpha + \beta) = \sigma (\alpha) + \sigma (\beta),\ \sigma (k\alpha)=k\sigma (\alpha),\ for\ any\ \alpha\ and\ \beta\in V_{1},k\in \ F.$$
\end{defi}

The necessary and sufficient condition for a Linear Mapping is 
$$\sigma (k\alpha +l\beta) = k\sigma(\alpha)+l\sigma(\beta).$$

\subsection{Identity Mapping}
Set $V$ as a vector space on the field $F$, a mapping $$\epsilon:V\rightarrow V$$
is defined as $\epsilon (\alpha)=\alpha,\ for\ all\ \alpha\in V.$
\subsection{Zero Mapping}
Set the $V_{1}$ and $V_{2}$ as vector spaces on the field $F$.
A mapping $$\tau :V_{1}\rightarrow V_{2}$$ 
is defined as $\tau (0)=0,\ for\ all\ \alpha\in V_{1}.$

\subsection{Properties}
If $\tau$ is a linear mapping, then it has follow properties:
\begin{pro}
    $\tau (\theta) = \theta,\ \tau(-\alpha)=-\tau(\alpha)$
\end{pro}
\begin{pro}
    Linear Mappings keep the linear combination and linear coefficents unchanged.
\end{pro}
\begin{pro}
    Linear Mappings transform the linear relative vector group into 
    another linear relative groups.
\end{pro}

\subsection{Matrix Reprentation of the Linear Mapping}
\begin{defi}
Set $\sigma$ as a linear mapping from $V_{1}$ to $V_{2}$, 
choose a basis $\alpha_{1},\alpha_{2},\cdots,\alpha_{n}$ in the $V_{1}$
and choose a basis $\beta_{1},\beta_{2},\cdots,\beta_{m}$ in the $V_{2}$.
If the image of the basis $\alpha_{1},\alpha_{2},\cdots,\alpha_{n}$ is
$$\begin{cases}
    \sigma(\alpha_{1})=a_{11}\beta_{1}+a_{21}\beta_{2}+\cdots+a_{m1}\beta_{m}\\
    \sigma(\alpha_{2})=a_{12}\beta_{1}+a_{22}\beta_{2}+\cdots+a_{m2}\beta_{m}\\
    \cdots\\
    \sigma(\alpha_{n})=a_{1n}\beta_{1}+a_{2n}\beta_{2}+\cdots+a_{mn}\beta_{m}
\end{cases}$$ 
and can be expressed as 
$$[\sigma(\alpha_{1}),\sigma(\alpha_{2}),\cdots,\sigma(\alpha_{n})]
=[\beta_{1},\beta_{2},\cdots,\beta_{m}]A.$$
where 
$A
=
\begin{bmatrix}
    a_{11}&a_{12}&\cdots&a_{1n}\\
    a_{21}&a_{22}&\cdots&a_{2n}\\
    \vdots&\vdots&\ddots&\vdots\\
    a_{m1}&a_{m2}&\cdots&a_{mn}
\end{bmatrix}$
 is called the linear mapping matrix of $\sigma$ under 
 the basis $\alpha$ and $\beta$.
\end{defi}

\begin{theorem}
    If $\sigma$ is a linear mapping from $V_{1}$ to $V_{2}$, 
    take a basis $\alpha_{1},\alpha_{2},\cdots,\alpha_{n}$ in the $V_{1}$, 
    and take a basis $\beta_{1},\beta_{2},\cdots,\beta_{m}$ in the $V_{2}$,
    then the linear mapping matrix of $\sigma$ under the basis $\alpha$ and $\beta$
    is A.

    For every $\alpha \in V$, if the coordinate of $\alpha$ under the basis $\alpha$
    is $(x_{1},x_{2},\cdots,x_{n})^{T}$, then the coordinate of $\sigma(\alpha)$ under the basis $\beta$
    is $(y_{1},y_{2},\cdots,y_{n})^{T}.$ Then 
    $$
    \begin{bmatrix}
        y_{1}\\
        y_{2}\\
        \vdots\\
        y_{m}
    \end{bmatrix}
    =
    A
    \begin{bmatrix}
        x_{1}\\
        x_{2}\\
        \vdots\\
        x_{n}
    \end{bmatrix}
    .$$
\end{theorem}

\begin{proof}
    Since 
    $$
    [\sigma(\alpha_{1}),\sigma(\alpha_{2}),\cdots,\sigma(\alpha_{n})]
=[\beta_{1},\beta_{2},\cdots,\beta_{m}]A,
    $$
    $$
    \alpha = [\alpha_{1},\alpha_{2},\cdots,\alpha_{n}]
    \begin{bmatrix}
        x_{1}\\
        x_{2}\\
        \vdots\\
        x_{n}
    \end{bmatrix}
    ,\\
    \sigma(\alpha) = [\sigma(\beta_{1}),\sigma(\beta_{2}),\cdots,\sigma(\beta_{n})]
    \begin{bmatrix}
        y_{1}\\
        y_{2}\\
        \vdots\\
        y_{n}
    \end{bmatrix},
    $$
    and 
    \begin{align*}
        \sigma{\alpha} 
        & =
        \sigma(
            [\alpha_{1},\alpha_{2},\cdots,\alpha_{n}]
            \begin{bmatrix}
                x_{1}\\
                x_{2}\\
                \vdots\\
                x_{n}
            \end{bmatrix}
        )\\
        & =
        [\sigma(\alpha_{1},\alpha_{2},\cdots,\alpha_{n})]
        \begin{bmatrix}
            x_{1}\\
            x_{2}\\
            \vdots\\
            x_{n}
        \end{bmatrix}
        & =
        [\beta_{1},\beta_{2},\cdots,\beta_{n}]A
        \begin{bmatrix}
            x_{1}\\
            x_{2}\\
            \vdots\\
            x_{n}
        \end{bmatrix}
    \end{align*}
    Then we hava
    $$
    \begin{bmatrix}
        y_{1}\\
        y_{2}\\
        \vdots\\
        y_{n}
    \end{bmatrix}
    =
    A
    \begin{bmatrix}
        x_{1}\\
        x_{2}\\
        \vdots\\
        x_{n}
    \end{bmatrix}
    .$$
\end{proof}

\begin{theorem}
    Set $\sigma$ as a linear transformation from $V_{1}$ to 
    $V_{2}$, $\alpha_{1},\alpha_{2},\cdots,\alpha_{n}$ and 
    $\alpha_{1}^{'},\alpha_{2}^{'},\cdots,\alpha_{n}^{'}$ 
    which 
    $$[\alpha_{1}^{'},\alpha_{2}^{'},\cdots,\alpha_{n}^{'}]=[\alpha_{1},\alpha_{2},\cdots,\alpha_{n}]P,$$
    are two basis of $V_{1}$, and $[\beta_{1}^{'},\beta_{2}^{'},\cdots,\beta_{n}^{'}]=
    [\beta_{1},\beta_{2},\cdots,\beta_{n}]Q.$

    If the linear mapping $\sigma$ under the basis $\alpha$ and $\beta$ is $A$, 
    under the basis $\alpha^{'}$ and $\beta^{'}$ is $B$, then 
    $$B=Q^{-1}AP.$$
\end{theorem}

\subsection{Isomorphism}
\begin{defi}
    Set $V_{1}$ and $V_{2}$ are linear spaces in the field $F$, if the 
    mapping $\sigma:V_{1}\rightarrow V_{2}$ has the follow properties:
    \begin{enumerate}
        \item $\sigma$ is a bijection.
        \item $\sigma (\alpha +\beta)=\sigma (\alpha)+\sigma(\beta),\ for\ all\ \alpha,\beta \in V_{1}$
        \item $\sigma(k\alpha)=k\sigma(\alpha),\ for\ all\ k\in F,\ all\ \alpha\in V_{1}$
    \end{enumerate}
    Then we say that $V_{1}$ and $V_{2}$ are isomorphic, and $\sigma$ is a 
    Isomorphism from $V_{1}$ to $V_{2}$, we denote $V_{1}\cong V_{2}.$
\end{defi}

\section{Linear Transformation}
\subsection{Linear Transformation}
\begin{defi}
    Set $\sigma$ as a transformation in linear space $V$ on the field $F$. 
    If for every $\alpha,\beta \in V,\ k \in F,$ that
    $$\sigma(\alpha+\beta)=\sigma(\alpha)+\sigma(\beta),\sigma(k\alpha)=k\sigma(\alpha)$$
    then we say $\sigma$ is a linear transformation on $V$. 
\end{defi}

\begin{example}[Scalar Multiplication]
    Set $k\in F$, define a transformation in $V$,
    $$\sigma (\alpha)=k \alpha, \alpha\in V$$
    It's easy to prove $\sigma$ is a linear transformation in $V$, 
    we call it a \textit{scalar multiplication transformation}.
\end{example}

Actualy, if $k=0$, this is a zero transformation, and if $k=1$, it is aidentity transformation.

\begin{defi}[Sum, Multiplication and Scalar multiplication of Transformation]
    Set $\sigma$ and $\tau$ as two transformations in linear space $V$, 
    $k\in F$, then the sum of $\sigma\ and\ \tau$, is defined as 
    $$(\sigma+\tau)(\alpha)=\sigma(\alpha)+\sigma(\beta),\forall \alpha\in V$$
    the multiplication of $\sigma$ and $\tau$, is defined as 
    $$(\sigma\circ\tau)(\alpha)=\sigma(\tau(\alpha)),\forall \alpha\in V$$
    the scalar multiplication of $\sigma$ and $k$, is defined as
    $$(\sigma\circ k)(\alpha)=k\sigma(\alpha),\forall \alpha\in V.$$
\end{defi}

\begin{defi}[Invertible Transformation]
    Set $\sigma$ as a transformation in linear space $V$, 
    if there exists a transformation $\tau$, satisfying that 
    $$\sigma \tau=\tau\sigma=\epsilon$$
    then we say $\sigma$ is an invertible transformation, and the 
    $\tau$ is the inverse transformation of $\sigma$, denoted as $\tau=\sigma^{-1}.$
\end{defi}

\textbf{Note.} The transformation is invertible if and only if it is bijection and its inverse 
transformation is unique.

\begin{theorem}
    Invertible transformation's inverse transformation is also a linear transformation.
\end{theorem}
\begin{proof}
    Set $\sigma$ as a invertible linear transformation, $\sigma^{-1}$ is its inverse transformation.
    Choose $\alpha_{1},\alpha_{2}\in V,k\in F$, and let
    $$\sigma^{-1}(\alpha_{1})=\beta_{1},\sigma^{-1}(\alpha_{2})=\beta_{2},$$
    then 
    \begin{align*}
        & \sigma^{-1}(\alpha_{1}+\alpha_{2})\\
         = &\sigma^{-1}(\sigma(\beta_{1}+\beta_{2}))\\
         = &\beta_{1}+\beta_{2}
    \end{align*}
    and
    $$\sigma^{-1}(k\alpha_{1})=k\beta_{1}=k\sigma(\alpha_{1}).$$
    So $\sigma^{-1}$ is also linear.
\end{proof}

\begin{defi}
    Set $\sigma$ as a linear transformation from $V_{n}$ to $V_{n}$,
    choose a basis $\alpha_{1},\alpha_{2},\cdots,\alpha_{n}$, if the image 
    of this basis under the transformation is 
    $$\begin{cases}
        \sigma(\alpha_{1})=a_{11}\alpha_{1}+a_{12}\alpha_{2}+\cdots+a_{1n}\alpha_{n}\\
        \sigma(\alpha_{2})=a_{21}\alpha_{1}+a_{22}\alpha_{2}+\cdots+a_{2n}\alpha_{n}\\
        \vdots\\
        \sigma(\alpha_{n})=a_{n1}\alpha_{1}+a_{n2}\alpha_{2}+\cdots+a_{nn}\alpha_{n}
    \end{cases}$$
    and can be expressed as 
    $$[\sigma(\alpha_{1}),\sigma(\alpha_{2}),\cdots,\sigma(\alpha_{n})]=[\alpha_{1},\alpha_{2},\cdots,\alpha_{n}]A$$
    where 
    $$A=
    \begin{bmatrix}
        a_{11}&a_{21}&\cdots&a_{n1}\\
        a_{12}&a_{22}&\cdots&a_{n2}\\
        \vdots&\vdots&\ddots&\vdots\\
        a_{1n}&a_{2n}&\cdots&a_{nn}
    \end{bmatrix}$$
    is called the linear transformation matrix of $\sigma$ under
    the basis $\alpha$.
\end{defi}

\textbf{Note.} After the basis is setted, the lineaer transformation and the matrix reprentation is unique.

If $\sigma$ is a linear transformation in the vector space $F^{n}$, and 
$$\sigma(\alpha)=A\alpha,\ \alpha\in F^{n}$$
then $A$ is the matrix reprentation of $\sigma$ in $F^{n}$ 
under the basis $\epsilon_{1},\epsilon_{2},\cdots,\epsilon_{n}$.

\begin{theorem}
    Set $\alpha_{1},\alpha_{2},\cdots,\alpha_{n}$ is a basis of the 
    linear vector $V_{n}$ on the field $F$, and two martrix reprentations of linear transformation 
    $\sigma$ and $\tau$ is $A$ and $B$, then
    \begin{enumerate}
        \item $\sigma+\tau$ under the basis is $A+B$.
        \item $\sigma\tau$ under the basis is $AB$.
        \item $\sigma(k\alpha)$ under the basis is $kA$.
        \item $\sigma$ is invertible if and only if $A$ is invertible, and 
        $\sigma^{-1}$ under the basis is $A^{-1}$.
    \end{enumerate}
\end{theorem}

\begin{theorem}[Matrix Transformation under the basis transformation]
    In the vector space $V_{n}$, we choose two basis
    $$\alpha_{1},\alpha_{2},\cdots,\alpha_{n};\beta_{1},\beta_{2},\cdots,\beta_{n},$$
    and the transition matrix from the basis $\alpha$ to the basis $\beta$ is $P$.
    Assume that the matrix reprentation of linear transformation $\sigma$ under the two basis 
    is $A$ and $B$, then 
    $$B=P^{-1}AP.$$
\label{MatrixTransUnderBasisTrans}
\end{theorem}

\begin{proof}
    For 
    \begin{align*}
        (\beta_{1},\beta_{2},\cdots,\beta_{n}) & = (\alpha_{1},\alpha_{2},\cdots,\alpha_{n})P\\
        \sigma[\alpha_{1},\alpha_{2},\cdots,\alpha_{n}] & = [\alpha_{1},\alpha_{2},\cdots,\alpha_{n}]A\\
        \sigma[\beta_{1},\beta_{2},\cdots,\beta_{n}] & = [\beta_{1},\beta_{2},\cdots,\beta_{n}]B\\
        then \\
        & [\beta_{1},\beta_{2},\cdots,\beta_{n}]B\\
         =&\sigma[\beta_{1},\beta_{2},\cdots,\beta_{n}]\\
         =&\sigma[(\alpha_{1},\alpha_{2},\cdots,\alpha_{n})P]\\
         =&\sigma[\alpha_{1},\alpha_{2},\cdots,\alpha_{n}]P\\
         =&[\alpha_{1},\alpha_{2},\cdots,\alpha_{n}]AP\\
         =&[\beta_{1},\beta_{2},\cdots,\beta_{n}]P^{-1}AP\\
    \end{align*}
    Since $\beta_{1},\beta_{2},\cdots,\beta_{n}$ is linear irrelative, 
    thus $$B=P^{-1}AP.$$
\end{proof}

\subsection{Eigenvalue and Eigenvector of the Linear Transformation}

\begin{defi}
    Set $\sigma$ as a linear transformation in the vector space $V$,
    if for a number $\lambda\in F$, there exists a \textbf{nonvector} $\alpha\in V$,
    that 
    $$\sigma(\alpha)=\lambda\alpha.$$
    then we say $\lambda$ is an \textit{eigenvalue} of linear transformation $\sigma$, and
    $\alpha$ is an \textit{eigenvector} of $\sigma$.
\end{defi}

Assume $\sigma$ is a linear transformation in the vector space $V$, and $\lambda$ is an eigenvalue 
of $\sigma$, then $\alpha$ is an eigenvector of $\sigma$ with eigenvalue $\lambda$, then $\sigma(\alpha)=\lambda \alpha.$
Choose a basis $\alpha_{1},\alpha_{2},\cdots,\alpha_{n}$ of $V$, and set 
$$\alpha=x_{1}\alpha_{1}+x_{2}\alpha_{2}+\cdots+x_{n}\alpha_{n}=\begin{bmatrix}\alpha_{1}&\alpha_{2}&\cdots&\alpha_{n}\end{bmatrix}
\begin{bmatrix}
    x_{1}\\
    x_{2}\\
    \vdots\\
    x_{n}
\end{bmatrix}.$$
then 
$$\sigma(\alpha)=\lambda\alpha=\lambda x_{1}\alpha_{1}+\lambda x_{2}\alpha_{2}+\cdots+\lambda x_{n}\alpha_{n}$$
and
$$\sigma(\alpha)=\begin{bmatrix}\alpha_{1}&\alpha_{2}&\cdots&\alpha_{n}\end{bmatrix}
(\lambda
\begin{bmatrix}
    x_{1}\\
    x_{2}\\
    \vdots\\
    x_{n}
\end{bmatrix})$$
then 
$$\lambda\begin{bmatrix}
    x_{1}\\
    x_{2}\\
    \vdots\\
    x_{n}
\end{bmatrix}
=
A\begin{bmatrix}
    x_{1}\\
    x_{2}\\
    \vdots\\
    x_{n}
\end{bmatrix}$$
where $A$ is the matrix reprentation of $\sigma$ under the basis $\alpha_{1},\alpha_{2},\cdots,\alpha_{n}$.
Let $X=(x_{1},x_{2},\cdots,x_{n})^{T}$, then 
$$AX=\lambda X.$$

Thus $\lambda$ is the eigenvalue of the matrix $A$, and the $X$ is the 
eigenvector of $A$ with eigenbvalue $\lambda$.

Vise versa, if $\lambda$ is the eigenvalue of the matrix $A$, and $X$ is the eigenvector of $A$ with eigenbvalue $\lambda$,
 then set $\alpha=x_{1}\alpha_{1}+x_{2}\alpha_{2}+\cdots+x_{n}\alpha_{n}$,
If $\alpha\in V$, and $\alpha\neq \theta$.
\begin{align*}
    \lambda\alpha&=\lambda([\alpha_{1},\alpha_{2},\cdots,\alpha_{n}]\begin{bmatrix}x_{1}\\x_{2}\\\vdots\\x_{n}\end{bmatrix})
    =[\alpha_{1},\alpha_{2},\cdots,\alpha_{n}](\lambda\begin{bmatrix}x_{1}\\x_{2}\\\vdots\\x_{n}\end{bmatrix})\\
    &=[\alpha_{1},\alpha_{2},\cdots,\alpha_{n}](\lambda X)=[\alpha_{1},\alpha_{2},\cdots,\alpha_{n}](AX)\\
    &=([\alpha_{1},\alpha_{2},\cdots,\alpha_{n}]A)X=\sigma([\alpha_{1},\alpha_{2},\cdots,\alpha_{n}]\begin{bmatrix}x_{1}\\x_{2}\\\vdots\\x_{n}\end{bmatrix})\\
    &=\sigma(\alpha)
\end{align*}
then $\lambda$ is the eigenvalue of $\sigma$ with eigenvector $\alpha$.



















\section{Exercise}

\begin{exe}
    Set $D:R[x]_{n+1}\rightarrow R[x]_{n}$ as \textit{Derivative Map}, you 
    should find the matrix reprentation of $D$ under the basis 
    $1,x,x^{2},\cdots,x_{n}$ and $1,x,x^{2},\cdots,x_{n-1}$.
\end{exe}
\begin{sol}
    Set $f_{1}=1,f_{2}=x,\cdots,f_{n+1}=x^{n}$,
    then $$D(f_{1})=0,D(f_{2})=1,D(f_{3})=2x,\cdots,D(f_{n+1})=nx^{n-1}.$$
    \begin{align*}
        &\begin{cases}
            D(f_{1})=0f_{1}+0f_{2}+0f_{3}+\cdots+0f_{n-1}\\
            D(f_{2})=1f_{1}+0f_{2}+0f_{3}+\cdots+0f_{n-1}\\
            D(f_{3})=0f_{1}+2f_{2}+0f_{3}+\cdots+0f_{n-1}\\
            \vdots\\
            D(f_{n+1})=0f_{1}+0f_{2}+\cdots+nf_{n-1}
        \end{cases}\\
        & [D(f_{1}),D(f_{2}),D(f_{3}),\cdots,D(f_{n+1})] \\
        & = [0,1,2x,\cdots,nx^{n-1}] \\ 
        & = [1,x,\cdots,x^{n-1}]
        \begin{bmatrix}
            0 & 1 & 0 & \cdots & 0\\
            0 & 0 & 2 & \cdots & 0\\
            \vdots & \vdots & \vdots & \ddots & \vdots\\
            0 & 0 & 0 & \cdots & n
        \end{bmatrix}
    \end{align*}

    Thus, the matrix representation of $D$ under the basis $1,x,x^{2},\cdots,x^{n}$ 
    and basis $1,x,x^{2},\cdots,x^{n-1}$ is
    $$
    \begin{bmatrix}
        0 & 1 & 0 & \cdots & 0\\
        0 & 0 & 2 & \cdots & 0\\
        \vdots & \vdots & \vdots & \ddots & \vdots\\
        0 & 0 & 0 & \cdots & n
    \end{bmatrix}
    $$
\end{sol}

\begin{exe}
    In $R^{3}$, we form a mapping $\sigma:R^{3}\rightarrow R^{3}$ 
    by $\sigma[(x_{1},x_{2},x_{3})]=(x_3,0,x_{2}-2x_{1}),\ (x_{1},x_{2},x_{3})\in R$.
    \begin{enumerate}
        \item Prove $\sigma$ is a linear mapping.
        \item Find the matrix representation of $\sigma$ under the basis
        $(1,0,0),(1,1,0),(1,1,1)$.
    \end{enumerate}
\end{exe}
\begin{sol}
    Choose any $(x_{1},x_{2},x_{3})$ and $(y_{1},y_{2},y_{3})$ $\in$ $R^{3}$, $k\in R$,
Since 
\begin{align*}
    \sigma[(x_{1},x_{2},x_{3})+(y_{1},y_{2},y_{3})]
    & = \sigma[(x_{1}+y_{1},x_{2}+y_{2},x_{3}+y_{3})]\\ 
    & = (x_{3}+y_{3},0,x_{2}+y_{2}-2(x_{1}+y_{1}))\\
    & =(x_{3},0,x_{2}-2x_{1})+(y_{3},0,y_{2}-2y_{1})\\
    & =\sigma[(x_{1},x_{2},x_{3})]+ \sigma[(y_{1},y_{2},y_{3})]\\
    \sigma[k(x_{1},x_{2},x_{3})]
    & =\sigma[(kx_{1},kx_{2},kx_{3})]\\
    & =(kx_{3},0,kx_{2}-2kx_{1})\\
    & =k(x_{3},0,x_{2}-2x_{1})\\
    & =k\sigma[(x_{1},x_{2},x_{3})]
\end{align*}
\end{sol}
\begin{sol}
Choose the natrual basis of $R^{3}$, $(1,0,0),(0,1,0),(0,0,1)$.
\begin{align*}
    \sigma[(1,0,0)]=(0,0,-2)\\
    \sigma[(0,1,0)]=(0,0,1)\\
    \sigma[(0,0,1)]=(1,0,0)
\end{align*}
and 
$$
\begin{cases}
    (0,0,-2)& =a_{11}(1,0,0)+a_{12}(1,1,0)+a_{13}(1,1,1)\\
    (0,0,1) & =a_{21}(1,0,0)+a_{22}(1,1,0)+a_{23}(1,1,1)\\
    (1,0,0) & =a_{31}(1,0,0)+a_{32}(1,1,0)+a_{33}(1,1,1)
\end{cases}
$$
$$
\begin{cases}
    a_{11}=0 ,\ a_{12}=2 ,\ a_{13}=-2\\
    a_{21}=0,\ a_{22}=-1,\ a_{23}= 1\\
    a_{31}=1,\ a_{32}=0,\ a_{33}=0
\end{cases}
$$
$$
\begin{bmatrix}
    0 & 0 & 1\\
    2 & -1 & 0\\
    -2 & 1 & 0
\end{bmatrix}
$$
That is the answer.
\end{sol}

\begin{exe}
    In the matrix space $R^{2\times2}$, we set a linear transformation $\sigma:$
    $\sigma(X)=AX,\ X\in R^{2\times2}$
    where
    $$
    A=\begin{bmatrix}
        1 & 2\\
        4 & 3
    \end{bmatrix}$$
    Find the representation of $\sigma$ under the natural basis in $R^{2\times2}$.
\end{exe}
\begin{sol}
    Set 
    $$I_{11}=\begin{bmatrix}
        1 & 0\\
        0 & 0
    \end{bmatrix},I_{12}=\begin{bmatrix}
        0 & 0\\
        0 & 1
    \end{bmatrix},I_{21}=\begin{bmatrix}
        0 & 1\\
        0 & 0
    \end{bmatrix},I_{22}=\begin{bmatrix}
        0 & 0\\
        0 & 0
    \end{bmatrix}$$
    and
    $$
    \begin{cases}
        \sigma(I_{11})=\begin{bmatrix}
                            1 & 0\\
                            4 & 0
                        \end{bmatrix}\\
        \sigma(I_{12})=\begin{bmatrix}
                            0 & 1\\
                            0 & 4
                        \end{bmatrix}\\
        \sigma(I_{21})=\begin{bmatrix}
                            2 & 0\\
                            3 & 0
                        \end{bmatrix}\\
        \sigma(I_{22})=\begin{bmatrix}
                            0 & 2\\
                            0 & 3
                        \end{bmatrix}
    \end{cases}
    $$
    Thus,
    $$
    \begin{cases}
        \sigma(I_{11})=1I_{11}+0I_{12}+4I_{21}+0I_{22}\\
        \sigma(I_{12})=0I_{11}+1I_{12}+0I_{21}+4I_{22}\\
        \sigma(I_{21})=2I_{11}+0I_{12}+3I_{21}+0I_{22}\\
        \sigma(I_{22})=0I_{11}+2I_{12}+0I_{21}+3I_{22}
    \end{cases}
    $$
    then 
    $$
    \begin{bmatrix}
        1 & 0 & 2 & 0\\
        0 & 1 & 0 & 2\\
        4 & 0 & 3 & 0\\
        0 & 4 & 0 & 3
    \end{bmatrix}
    $$
    is the answer.
\end{sol}

\begin{exe}
    We know that a linear transformation $\sigma$'s matrix reprentation under the 
    basis $\alpha_{1},\alpha_{2},\alpha_{3}$ is 
    $$ A=
    \begin{bmatrix}
        1&2&3\\
        4&5&6\\
        7&8&9
    \end{bmatrix}.$$
    Find the matrix reprentation of $\sigma$ under the basis $\alpha_{3},\alpha_{2},\alpha_{1}$
\end{exe}
\begin{sol}
    Since
    $$
    [\alpha_{3},\alpha_{2},\alpha_{1}]
    =[\alpha_{1},\alpha_{2},\alpha_{3}]
    \begin{bmatrix}
        0&0&1\\
        0&1&0\\
        1&0&0
    \end{bmatrix}
    ,$$
    then from the theorem \ref{MatrixTransUnderBasisTrans}, 
    \begin{align*}
    B
    & =
    \begin{bmatrix}
        0&0&1\\
        0&1&0\\
        1&0&0
    \end{bmatrix}
    A
    \begin{bmatrix}
        0&0&1\\
        0&1&0\\
        1&0&0
    \end{bmatrix}\\
    &=
    \begin{bmatrix}
        7&8&9\\
        4&5&6\\
        1&2&3
    \end{bmatrix}
    \begin{bmatrix}
        0&0&1\\
        0&1&0\\
        1&0&0
    \end{bmatrix}\\
    &=
    \begin{bmatrix}
        9&8&7\\
        6&5&4\\
        3&2&1
    \end{bmatrix}
    \end{align*}
    is the answer.
\end{sol}

\begin{exe}
    Set the linear transformation $\sigma$ under the basis 
    $\alpha_{1},\alpha_{2},\alpha_{3}$, and its matrix reprentation is 
    $$A=
    \begin{bmatrix} 
        1&2&2\\
        2&1&2\\
        2&2&1
    \end{bmatrix}.$$
    Find the eigenvalue of $\sigma$ and its eigenvector.
\end{exe}

\begin{sol}
    Solve the eigenvalue of the $A$,
    \begin{align*}
    &|\lambda I-A|\\
    &=
    \begin{vmatrix}
        \lambda-1&-2&-2\\
        -2&\lambda-1&-2\\
        -2&-2&\lambda-1
    \end{vmatrix}\\
    &=
    \begin{vmatrix}
        \lambda-5&-2&-2\\
        \lambda-5&\lambda-1&-2\\
        \lambda-5&-2&\lambda-1
    \end{vmatrix}\\
    &=(\lambda-5)
    \begin{vmatrix}
        1&-2&-2\\
        1&\lambda-1&-2\\
        1&-2&\lambda-1
    \end{vmatrix}\\
    &=(\lambda-5)
    \begin{vmatrix}
        1&0&0\\
        1&\lambda+1&0\\
        1&0&\lambda+1
    \end{vmatrix}\\
    &=(\lambda+1)^{2}(\lambda-5)
    \end{align*}
    Thus, the eigenvalue of $\sigma$ is $\lambda_{1}=\lambda_{2}=-1$ or $\lambda_{3}=5$.
    For $\lambda=-1$,
    $$
    \begin{cases}
        -2x_{1}-2x_{2}-2x_{3}=0\\
        -2x_{1}-2x_{2}-2x_{3}=0\\
        -2x_{1}-2x_{2}-2x_{3}=0
    \end{cases}
    $$
    then two solutions of it is $X_{1}=(1,0,-1)^{T}$ and 
    $X_{2}=(0,1,-1)$.
    For $\lambda=5$,
    $$
    \begin{cases}
        4x_{1}-2x_{2}-2x_{3}=0\\
        -2x_{1}+4x_{2}-2x_{3}=0\\
        -2x_{1}-2x_{2}+4x_{3}=0
    \end{cases}
    $$
    then one solution of it is $X_{3}=(1,1,1)^{T}$.

    For eigenvectors with eigenvalue $\lambda=-1$,
    \begin{align*}
        \beta_{1}&=[\alpha_{1},\alpha_{2},\alpha_{3}]X_{1}=\alpha_{1}-\alpha_{3}\\
        \beta_{2}&=[\alpha_{1},\alpha_{2},\alpha_{3}]X_{2}=\alpha_{2}-\alpha_{3}
    \end{align*}
    then the eigenvector of the $\sigma$ with eigenvalue -1 is 
    $$k_{1}\beta_{1}+k_{2}\beta_{2}.$$
    
    For eigenvalue with eigenvalue $\lambda=5$,
    $$\beta_{3}=[\alpha_{1},\alpha_{2},\alpha_{3}]X_{3}=\alpha_{1}+\alpha_{2}+\alpha_{3}$$
    then $$k_{3}\beta_{3}$$ is the eigenvector of the $\sigma$ with eigenvalue 5.
\end{sol}


\section{Conclusion}
\begin{enumerate}
    \item Judge whether it's a linear transformation.
    \item Find the matrix representation of a linear transformation under the specific basis.
    \item Find the matrix representation of a linear transformation under the different basis.
\end{enumerate}




































\end{document}