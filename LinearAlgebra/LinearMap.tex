\documentclass[12pt]{article} % 可以选择不同的文档类型,例如 report 或 book
\usepackage[utf8]{inputenc}
\usepackage{amsmath}
\usepackage{graphicx}
\usepackage{hyperref}
\usepackage{ctex}
\usepackage{amsmath}
\usepackage{amsfonts}
\usepackage{amssymb}




\title{Linear Map} 
\author{Len Fu} 
\date{11.19.24} 

\begin{document}

% 显示标题
\maketitle

% 摘要
\begin{abstract}
This is the note of Linear Map, maded by Len Fu while his learning progress. 
The main content is from \textit{Linear Algebra Done Right} , \textit{线性代数\ 北京理工大学出版社}.
\ and \textit{Linear Algebra\  Allenby}
\end{abstract}

% 目录
\tableofcontents
\newpage

% 正文开始
\section{Definition and Examples} % 第一级标题
\subsection{Definiton}
A mapping\ $T$\ from a vector space\ $V$\ to a vector space\ $W$\ is called 
a \textit{linear transformation} or a \textit{linear operator} if,
$$T(\alpha v_{1}+\beta\ v_{2})=\alpha T(v_{1})+\beta T(v_{2})$$
for all\ $v_{1},v_{2}\in V$\ and all\ $\alpha,\beta\in\mathbf{F}$.\\
Then if\ $T$\ is a linear operator on $V$ if and only $T$ satisfies
$$T(v_{1}+v_{2})=T(v_{1})+T(v_{2})$$
and 
$$T(\alpha v_{1})=\alpha T(v_{1})$$.
Thus $T$ is a linear transformation mapping a vector space\ $V$\ to a vector space\ $W$ 
if and only if $T$ is a linear operator on $V$.\\

\subsection{Examples} % 第二级标题


\subsubsection{Linear Operators from $R^{n}$ to $R^{m}$} 
In general, if $A$ is any $m\times n$ matrix, we can define a linear operator $T_{A}$ 
from $R^{n}$ to $R^{m}$ $$T_{A}(x)=Ax$$ for all $x\in R^{n}$. The operator $T_{A}$ is linear, 
since
\begin{align}
T_{A}(\alpha x+\beta y)&=A(\alpha x+\beta y)\\
&=\alpha Ax+\beta Ay\\
&=\alpha T_{A}(x)+\beta T_{A}(y)
\end{align}
Thus we can think of each $m\times n$ matrix as defining a linear operator from $R^{n}$ to $R^{m}$.

\subsubsection{Linear Transformation from $V$ to $W$}



\section{The Matrix of a Linear Map}
Let $T\ \in\ \mathcal{L}(V,W)$. Suppose that\ $(v_{1},...,v_{n})$\ is a basis of\ $V$\ and 
$(w_{1},...,w_{m})$\ is a basis of\ $W$\ .For each $k\ =\ 1,...,n$, we can write\ $Tv_{k}$\ 
uniquely as a linear combination of\ $w_{1},...,w_{m}$: 
$$Tv_{k}\ =\ a_{1k}w_{1}\ +\ ...\ +\ a_{mk}w_{m},$$
where\ $a_{jk}\in\ \mathbf{F}\ for\ j\ in\ 1,...,m.$.\\
The matrix
$$
\begin{bmatrix}
a_{11} &  ... & a_{1n}\\
.& &.\\
.& &.\\
.& &.\\
a_{m1} & ... & a_{mn}
\end{bmatrix}
$$
is called the \textit{matrix}\ of\ $T$\ with respect to the 
bases\ $(v_{1},...,v_{n})$\ and\ $(w_{1},...,w_{m})$.
We denote it by 
$$\mathcal{M}(T,(v_{1},...,v_{n}),(w_{1},...,w_{m}))$$.
or we just write $\mathcal{M}(T)$. The\ $k^{th}$\ column consists of the 
scalars needed to write\ $Tv_{k}$\ as a linear combination of\ $w_{1},...,w_{m}$. \\

\section{Changing coordinate systems}



% 结论
\section{Conclusion}
This is the conclusion section. Summarize the findings and state any implications or future work.

% 参考文献
\bibliographystyle{plain} % 参考文献样式
\bibliography{references} % 参考文献文件

\end{document}