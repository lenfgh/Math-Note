\documentclass{article}
\usepackage[left=2cm,right=2cm,top=2cm,bottom=2cm]{geometry}
\usepackage{amsmath}
\usepackage{hyperref}
\usepackage{amssymb}
\usepackage{enumitem}
\usepackage{setspace}
\usepackage{amsthm}

\newtheorem{theorem}{Theorem}[section]
\newtheorem{lemma}{Lemma}[theorem]

\theoremstyle{definition} 
\newtheorem{defi}{Definition}[section]
\newtheorem{exa}{Example}[defi]
\newtheorem{exe}{Exercise}[section]
\newtheorem{sol}{Solution}[exe]
\newtheorem{pro}{Properties}[section]


\linespread{1.5}


\title{Eigenvalue and Eigenvector}
\author{Len Fu}
\date{11.27.2024}

\begin{document}

\maketitle

\begin{abstract}
    This is the note made by Len Fu during his learning progress in BIT.
    The main content is from \textit{Linear Algebra Done Right} and \textit{Linear Algebra Allenby}.
\end{abstract}

\tableofcontents

\newpage

\section{Similarity of the Matrix}
\subsection{Basis}
\begin{defi}
    Set $A,B\in C^{n\times n}$. If there exists
    an n-order invertible matrix $P$ such that 
    $$P^{-1}AP=B$$, we say that $A$ and $B$ are similar,
    denoted as $A\sim B$, and $P$ is called the 
    \textit{similarity transformation} from $A$ to $B$. 
\end{defi}

\begin{pro}[Reflectivity]
    $A\sim A$.
\end{pro}

\begin{pro}[Symmetry]
    If $A\sim B$, then $B\sim A$.
\end{pro}

\begin{pro}[Transitivity]
    If $A\sim B$ and $B\sim C$, then $A\sim C$.
\end{pro}

\subsection{Similar Diagonalization}
\begin{defi}[Digonalizable]
    If there exists an invertible matrix $P$ such that
    $$P^{-1}AP=D$$
    where $A$ is a square and $D$ is a diagonal matrix.
    Then $A$ is called \textit{diagonalizable}.
\end{defi}


\section{Eigenvalue and Eigenvector of the Matrix}
\subsection{Basis}\
\begin{defi}
    Set $A$ as a $n\times n$ square, if there exists a 
    number $\lambda$ and $n-nonzero$ vector $X$, satisfying 
    $$AX=\lambda X\ or\ (\lambda I-A)X=0$$
    then we say that $\lambda$ is an eigenvalue of $A$, and
    $X$ is an eigenvector of $A$ with eigenvalue $\lambda$.
\end{defi}

\textbf{Note.}
\begin{enumerate}
    \item Only squares have eigenvectors and eigenbvalues.
    \item Eigenvector must be nonvector and eigenvalue can be zero.
\end{enumerate}

Since $(\lambda I-A)X=0$ and $X$ is nonzero vector, then 
$\det (\lambda I-A)$ should be zero to ensure $X$ is nonzero vector of the solution.

Consider the solution of $(\lambda I-A)X=0$.
The characteristic polynomial of $A$ is 
$$b_{n}\lambda^{n}+b_{n-1}\lambda^{n-1}+\cdots+b_{1}\lambda+b_{0}.$$

To solve the polynomial,
\begin{align*}
    &\begin{vmatrix}
        \lambda-a_{11} & -a_{12} & \cdots & -a_{1n} \\
        -a_{21} & \lambda-a_{22} & \cdots & -a_{2n} \\
        \vdots & \vdots & \ddots & \vdots \\
        -a_{n1} & -a_{n2} & \cdots & \lambda-a_{nn}
    \end{vmatrix}
    \\ & =
    b_{n}\lambda^{n}+b_{n-1}\lambda^{n-1}+\cdots+b_{1}\lambda+b_{0}
\end{align*}
Consider the expasion of the determinant, except for 
$$(\lambda-a_{11})(\lambda-a_{22})\cdots (\lambda-a_{nn})$$ 
other terms' highest order of $\lambda$ is $n-2$.
Then the coefficents
$$\begin{cases}
    b_{n} = 1 \\
    b_{n-1} = -(a_{11}+a_{22}+\cdots+a_{nn}) = tr()\\ 
\end{cases}
$$





\end{document}